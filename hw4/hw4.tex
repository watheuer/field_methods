\documentclass[12pt]{article}

% packages
\usepackage{tipa}
\usepackage{enumerate}
\usepackage{linguex}
\usepackage{gb4e}
\noautomath
\usepackage{xstring}
\usepackage{multirow}
\usepackage{footnote}
\makesavenoteenv{tabular}
\makesavenoteenv{table}

% define \phon for phonetic examples
\newcommand{\phon}[1]{$[$\textipa{#1}$]$}
\newcommand{\orth}[1]{\textit{\StrSubstitute{#1}{I}{\'{i}}[\x]\StrSubstitute{\x}{E}{\'{e}}[\x]\StrSubstitute{\x}{N}{\~{n}}[\x]\x}}

% margins and spacing
\usepackage[letterpaper, margin=1in]{geometry}

\iffalse
Possession: 

We've seen two patterns for possession: ijé 'my hand' and yené ij 'my hand'. Name the two patterns and describe their uses (what is the difference between the two patterns? are both used for alienable and inalienable possession?). Make sure you include the different persons (with examples--not just a chart) and any variation in the form of affixes. Remember that possession is not just forms meaning 'my', 'your', etc.: you also need to include forms like 'Rachel's hand'.

You don't need to describe the use of -u and -wa for definiteness right now.
\fi


\begin{document}

\begin{center}
{\Large Homework 4} \\
{\large Will Theuer}
\end{center}

\section{Possession}

There are two different strategies for possession in Amharic, both of which are used for alienable and inalienable possession.

\subsection{Suffixes}

Possession is marked with noun suffixes. Second and third person singular suffixes distinguish between male and femaile, but the plural forms are not gendered.

\begin{exe}
  \ex \orth{doro-E} `my chicken'\\
      \orth{doro-i} `your (m) chicken'\\
      \orth{doro-ish} `your (f) chicken'\\
      \orth{doro-u} `his chicken'\\
      \orth{doro-wa} `her chicken'
\end{exe}

\noindent The second person masculine suffix also occurs as \orth{-ih}.
\begin{exe}
  \ex\label{ex:male_suffix_h} \orth{ij-ih} `your (m) hand'
\end{exe}

All plural suffixes begin with \orth{ach-}. These suffixes may be further analyzable as \orth{-ach-in, -ach-un, and -ach-ew}.
\begin{exe}
  \ex \orth{doro-achin} `our chicken' \\
      \orth{doro-achun} `your (pl) chicken'\\
      \orth{doro-achew} `their chicken'
\end{exe}

\subsection{Contrast forms}

Additionally, Amharic has several contrast forms for possession. These forms place the stress on the possessor and convey a meaning like `MY hand.' Each of these can also be expressed with a suffix form as in the previous section.

\begin{table}[ht]
\centering
\caption{Contrast forms}
\label{tab:contrast}
  \begin{tabular}{llll}
    \orth{yenE ij}      & `my hand'             & \orth{ij-E} & `my hand' \\
    \orth{yante ij}     & `your (m) hand'       & \orth{ij-ih} & `your (m) hand'\\
    \orth{yanchI ij}    & `your (f) hand'       & \orth{ij-ish} & `your (f) hand'\\
    \orth{yesu ij}      & `his hand'            & \orth{ij-u} & `his hand'\\
    \orth{yeswa ij}     & `her hand'            & \orth{ij-wa} & `her hand'\\
    \orth{yenya ij-och} & `our hands'           & \orth{ij-och-achin} & `our hands'\\
    -\footnote{There was no example of a contrast form for the second person plural (you all).} & - & \orth{ij-och-achun} & `your (pl) hands'  \\ 
    \orth{yenesu ij-och} & `their hands' & \orth{ij-och-achew} & `their hands' \\
 \end{tabular}
\end{table}

Possessive forms like `Rachel's hand' are constructed similarly. As shown in (\ref{ex:bad_possession}), these constructions do not work with the suffix forms in the previous section.

\begin{exe}
  \ex\label{ex:possession_female} \orth{je Rachel ij} `Rachel's hand'
  \ex\label{ex:bad_possession} *\orth{Rachel ijwa} `Rachel's hand'
  \ex\label{ex:possession_male} \orth{je Abe ij} `Abe's hand'
\end{exe}

\noindent We can see in (\ref{ex:possession_female}) and (\ref{ex:possession_male}) that these forms are the same for male and female.

\section{Pronouns}

\iffalse
Pronouns:
What types of pronouns are there and what names will you give them? So far I see words like iné  'I', ihé 'this', and yené 'mine'. I would probably list interrogative pronouns in a chapter on questions.
Grading rubric: Did you describe all the patterns in the data in a logical, well-ordered way? Do you rise above details to give a good feel for how the language works? Did you include enough data to document the language? Is your data accurate? Are you using normal format and terms for describing a language? Do you define your terms? Is your writing carefully edited?
\fi

Subject pronouns have the following forms in Amharic:

\begin{table}[ht]
\centering
\caption{Subject pronouns}
\label{tab:pronouns}
  \begin{tabular}{ll}
    \orth{inE} & I \\
    \orth{ante} & you (m) \\
    \orth{anchI} & you (f) \\
    \orth{isu} & he \\
    \orth{iswa} & she \\
    \orth{inya} & we \\
    \orth{inante} & you (pl) \\
    \orth{inesu} & they \\
  \end{tabular}
\end{table}

These pronouns only occur as the subjects of sentences, and they occur before objects and verbs. This provides some evidence that suggests that Amharic is an SOV language.

\begin{exe}
  \ex \gll \orth{inE} \orth{rejim} \orth{neny} \\
  I tall be\textbackslash\textsc{1s} \\
  \trans  `I am tall.'

  \ex \gll \orth{ante} \orth{rejim} \orth{neh}\\
  you-\textsc{m} tall be\textbackslash\textsc{2ms}\\
  \trans `You (m) are tall'

  \ex \gll \orth{anchI} \orth{rejim} \orth{nesh}\\
  you-\textsc{f} tall be\textbackslash\textsc{2fs}\\
  \trans `You (f) are tall'

  \ex \gll \orth{isu} \orth{rejim} \orth{new}\\
  he tall be\textbackslash\textsc{3ms}\\
  \trans `He is tall'

  \ex \gll \orth{iswa} \orth{rejim} \orth{nat}\\
  she tall be\textbackslash\textsc{3fs}\\
  \trans `She is tall'

  \ex \gll \orth{inya} \orth{rejim} \orth{nen}\\
  we tall be\textbackslash\textsc{1p}\\
  \trans `We are tall'

  \ex \gll \orth{inant} \orth{rejim} \orth{nachu}\\
  you\textbackslash\textsc{p} tall be\textbackslash\textsc{2p}\\
  \trans `You (pl) are tall'


  \ex \gll \orth{inesu} \orth{rejim} \orth{nachew}\\
  they tall be\textbackslash\textsc{3p}\\
  \trans `They are tall'
\end{exe}


Additionally, the language has demonstrative pronouns (like `this' and `that'). These pronouns appear to be grammatically masculine; in (\ref{ex:this}), we see that the verb \orth{new} `to be' is in the masculine form, even though \orth{wef} `bird' is grammatically feminine. 

\begin{exe}
  \ex\label{ex:this} \gll \orth{ihE} \orth{k'ey} \orth{wef} \orth{new} \\
  this red bird be\textbackslash\textsc{3ms}\\
  \trans `This is a red bird'\footnote{In these glosses, \textsc{3fs} means third person feminine singular, \textsc{3pl} means third person plural, etc.} 

  \ex \gll \orth{inezI} \orth{k'ey} \orth{wef-och} \orth{nachew} \\
  these red bird-\textsc{p} be\textbackslash\textsc{3p}\\
  `These birds are red'

  \ex \gll \orth{yachI} \orth{wef} \orth{tinishIyE} \orth{nat} \\
  that bird small be\textbackslash\textsc{3fs} \\
  `That bird is small'
\end{exe}

There is evidence of possessive pronouns, but we only see a single example in the data.

\begin{exe}
  \ex \orth{yenE} 'mine'
\end{exe}

\end{document}


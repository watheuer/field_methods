\documentclass[12pt]{article}

% packages
\usepackage{tipa}
\usepackage{enumerate}
\usepackage{linguex}
\usepackage{gb4e}
\noautomath
\usepackage{xstring}
\usepackage{multirow}
\usepackage{multicol}
\usepackage{footnote}
\makesavenoteenv{tabular}
\makesavenoteenv{table}

% define \phon for phonetic examples
\newcommand{\phon}[1]{$[$\textipa{#1}$]$}
\newcommand{\orth}[1]{\textit{\StrSubstitute{#1}{I}{\'{i}}[\x]\StrSubstitute{\x}{E}{\'{e}}[\x]\StrSubstitute{\x}{N}{\~{n}}[\x]\x}}

% margins and spacing
\usepackage[letterpaper, margin=1in]{geometry}
% ==============================================================================================================

\begin{document}

\begin{center}
{\Large Homework 7} \\
{\large Will Theuer}
\end{center}

\section{Causatives}
\iffalse
1. We have seen a few examples of causatives in Amharic:

kiflén ats’edawt ‘I clean my room’
inaté kiflén asts’edachíny ‘My mom made me clean my room’
 
iné arigkut ‘I did it’
iswa asderigechíny ‘she made me do it’ [I don't know what this "de-" is]
 
iné k’omkuwin ‘I stopped’
polísu ask’omeny ‘the policeman caused me to stop’
Describe what is going on in these pairs of examples: what is happening to the verb? how is agreement working (what is the verb agreeing with)? what is getting case marked as an object? To make your assumptions clear, draw a tree for one non-causative sentence and one causative counterpart.
\fi

\begin{exe}
  \ex \orth{mats'dat} `to clean' \\
      \orth{kiflEn ats'edawt} `I clean my room' \\
      \orth{inatE kiflEn asts'edachIny} `My mom made me clean my room'

  \ex \orth{madrig} `to do' \\
      \orth{inE arigkut} `I did it' \\
      \orth{iswa asderigechIny} `She made me do it'
\end{exe}

\begin{exe}
  \ex \gll \orth{kiflEn} \orth{ats'edawt} \\
  room something \\
  \trans `I clean my room'
\end{exe}


\section{Reflexives}
\iffalse
2. Reflexives. We have seen reflexive examples like these (note: ras means 'head'):

sétochu inesun ayu
the women saw them
 
sétochu irasachewn ayu
the women saw themselves

irasény tat’ebkuwin
I washed myself
 
irasin tat’ebk
You m washed yourself
 
irasishin tat’ebsh
You f washed yourself
 
irasun tat’ebe
He washed himself
 
iraswan tat’ebech
she washed herself
 
irasachin tat’eben
we washed ourselves
 
irasachun tat’ebachu
you all washed yourselves
 
irasachewn tat’ebu
they washed themselves

Describe what is going on these examples.
\fi

\begin{exe}
  \ex \orth{mat'eb} `to wash' \\
  \begin{tabular}{lll}
    \orth{i-ras-Eny} & \orth{tat'eb-kuwin} & `I washed myself' \\
    \orth{i-ras-in} & \orth{tat'eb-k} & `You (m) washed yourself'\\
    \orth{i-ras-ishin} & \orth{tat'eb-sh} & `You (f) washed yourself'\\
    \orth{i-ras-un} & \orth{tat'eb-e} & `He washed himself'\\
    \orth{i-ras-wan} & \orth{tat'eb-ech} & `She washed herself'\\
    \orth{i-ras-achin} & \orth{tat'eb-en} & `We washed ourselves'\\
    \orth{i-ras-achun} & \orth{tat'eb-achu} & `You all washed yourselves'\\
    \orth{i-ras-achewn} & \orth{tat'eb-u} & `They washed themselves'
  \end{tabular}
\end{exe}

\begin{exe}
  \ex \gll \orth{i-ras-E-ny} \orth{tat'eb-kuwin} \\
  \textsc{refl}-head-\textsc{1s.poss}-\textsc{refl} wash-\textsc{1s} \\
  \trans `I washed myself'
\end{exe}

\end{document}

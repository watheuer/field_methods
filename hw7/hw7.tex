\documentclass[12pt]{article}

% packages
\usepackage{tipa}
\usepackage{enumerate}
\usepackage{linguex}
\usepackage{gb4e}
\noautomath
\usepackage{xstring}
\usepackage{multirow}
\usepackage{multicol}
\usepackage{footnote}
\usepackage{qtree}
\makesavenoteenv{tabular}
\makesavenoteenv{table}

% define \phon for phonetic examples
\newcommand{\phon}[1]{$[$\textipa{#1}$]$}
\newcommand{\orth}[1]{\textit{\StrSubstitute{#1}{I}{\'{i}}[\x]\StrSubstitute{\x}{E}{\'{e}}[\x]\StrSubstitute{\x}{N}{\~{n}}[\x]\x}}

% margins and spacing
\usepackage[letterpaper, margin=1in]{geometry}
% ==============================================================================================================

\begin{document}

\begin{center}
{\Large Homework 7} \\
{\large Will Theuer}
\end{center}

\section{Causatives}
\iffalse
1. We have seen a few examples of causatives in Amharic:

kiflén ats’edawt ‘I clean my room’
inaté kiflén asts’edachíny ‘My mom made me clean my room’
 
iné arigkut ‘I did it’
iswa asderigechíny ‘she made me do it’ [I don't know what this "de-" is]
 
iné k’omkuwin ‘I stopped’
polísu ask’omeny ‘the policeman caused me to stop’
Describe what is going on in these pairs of examples: what is happening to the verb? how is agreement working (what is the verb agreeing with)? what is getting case marked as an object? To make your assumptions clear, draw a tree for one non-causative sentence and one causative counterpart.
\fi

Causatives are marked with the verbal prefix \orth{as-}. In the causative form, the verb agrees with the causer (in the subject position) and the causee (in the object position). This is different from the non-causative form, in which the verb agrees with the subject and the object of the sentence, as seen in (\ref{ex:caus:clean1}) with the verb \orth{mats'dat} `to clean.' However, the object is marked with the object suffix \orth{-n} in both forms. We can see the causative form in (\ref{ex:caus:clean2}).

\iffalse
\begin{exe}
  \ex \orth{mats'dat} `to clean' \\
      \orth{kiflEn ats'edawt} `I clean my room' \\
      \orth{inatE kiflEn asts'edachIny} `My mom made me clean my room'

  \ex \orth{madrig} `to do' \\
      \orth{inE arigkut} `I did it' \\
      \orth{iswa asderigechIny} `She made me do it'
\end{exe}
\fi

\begin{exe}
  \ex\label{ex:caus:clean1} \gll \orth{kifl-E-n} \orth{ats'eda-w-t} \\
  room-\textsc{1s.poss-obj} clean-\textsc{1s.subj-3sm.obj} \\
  \trans `I cleaned my room'

  \ex\label{ex:caus:clean2} \gll \orth{inat-E} \orth{kifl-E-n} \orth{as-ts'eda-ch-iny} \\
  mom-\textsc{1s.poss} room-\textsc{1s.poss-obj} \textsc{caus}-clean-\textsc{3sf.subj-1s.obj} \\
  \trans `My mom made me clean my room'
\end{exe}

\noindent (\ref{ex:caus:tree1}) and (\ref{ex:caus:tree2}) show the structures of both sentences.

\begin{exe}
  \ex\label{ex:caus:tree1} \Tree [.S [.NP $\emptyset$ ] [.VP [.NP N\\\orth{kifl-E-n}\\room-\textsc{1s.poss-obj} ] V\\\orth{ats'eda-w-t}\\clean-\textsc{1s.subj-3sm.obj} ] ]

  \ex\label{ex:caus:tree2} \Tree [.S [.NP N\\\orth{inat-E}\\mom-\textsc{1s.poss} ] [.VP [.NP N\\\orth{kifl-E-n}\\room-\textsc{1s.poss-obj} ] V\\\orth{as-ts'eda-ch-iny}\\\textsc{caus}-clean-\textsc{3sf.subj-1s.obj} ] ]
\end{exe}

\section{Reflexives}
\iffalse
2. Reflexives. We have seen reflexive examples like these (note: ras means 'head'):

sétochu inesun ayu
the women saw them
 
sétochu irasachewn ayu
the women saw themselves

irasény tat’ebkuwin
I washed myself
 
irasin tat’ebk
You m washed yourself
 
irasishin tat’ebsh
You f washed yourself
 
irasun tat’ebe
He washed himself
 
iraswan tat’ebech
she washed herself
 
irasachin tat’eben
we washed ourselves
 
irasachun tat’ebachu
you all washed yourselves
 
irasachewn tat’ebu
they washed themselves

Describe what is going on these examples.
\fi

Reflexives are formed with the noun \orth{iras}, which may be derived from \orth{ras} `head.' \orth{iras} is the object of the sentence and takes a possessive and a \orth{-n/-ny} object suffix. In (\ref{ex:reflexive:wash}), we see the paradigm for the verb \orth{metat'eb} `to wash.'

\begin{exe}
  \ex\label{ex:reflexive:wash} \orth{metat'eb} `to wash' \\
  \begin{tabular}{lll}
    \orth{iras-E-ny} & \orth{tat'eb-kuwin} & `I washed myself' \\
    \orth{iras-i-n} & \orth{tat'eb-k} & `You (m) washed yourself'\\
    \orth{iras-ish-in} & \orth{tat'eb-sh} & `You (f) washed yourself'\\
    \orth{iras-u-n} & \orth{tat'eb-e} & `He washed himself'\\
    \orth{iras-wa-n} & \orth{tat'eb-ech} & `She washed herself'\\
    \orth{iras-achin} & \orth{tat'eb-en} & `We washed ourselves'\\
    \orth{iras-achun} & \orth{tat'eb-achu} & `You all washed yourselves'\\
    \orth{iras-achew-n} & \orth{tat'eb-u} & `They washed themselves'
  \end{tabular}
\end{exe}

Each of these forms of \orth{iras} has an added \orth{-n/-ny/-in} object suffix unless the possessive suffix already ends in \orth{n}\footnote{This is probably an object suffix that occurs throughout the language. This requires some more data to prove.}. The \orth{-ny} form occurs after the tense \orth{-E} suffix in the \textsc{1s} form, and the \orth{-in} form occurs after \orth{-ish}. This may be a result of the cluster \orth{*ishn} not being allowed in the language. (\ref{ex:reflexive:wash:gll1}) and (\ref{ex:reflexive:wash:gll2}) show the glosses of two of the forms.

\begin{exe}
  \ex\label{ex:reflexive:wash:gll1} \gll \orth{iras-E-ny} \orth{tat'eb-kuwin} \\
  \textsc{refl}-\textsc{1s.poss}-\textsc{obj} wash-\textsc{1s} \\
  \trans `I washed myself'

  \ex\label{ex:reflexive:wash:gll2} \gll \orth{iras-i-n} \orth{tat'eb-k} \\
  \textsc{refl}-\textsc{2sm.poss}-\textsc{obj} wash-\textsc{2sm} \\
  \trans `You (m) washed yourself'
\end{exe}

The verb form is conjugated as we expect from section ??\footnote{This will point to the actual section in the final write-up...} on subject person marking. It agrees with the subject of the sentence, which either occurs before the object or only as a verbal suffix (as seen here).


\end{document}

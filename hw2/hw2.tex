\documentclass[12pt]{article}
\usepackage[top=1in,bottom=1in,right=1in,left=1in]{geometry}
\usepackage{tipa}
\usepackage{gb4e}
\usepackage{enumerate}
\usepackage{amsmath}
\usepackage{setspace}
\usepackage{apacite}
\usepackage{url}

\newcommand{\ipa}[1]{$[$\textipa{#1}$]$}

\pagestyle{empty}
\doublespacing

\begin{document}

\begin{center}
{\Large HW2} \\
{\large Will Theuer } \\
\end{center}

\iffalse
\fi

\section{Phonology}

Amharic has several interesting features, including a three-way contrast for stops between voiced, voiceless, and ejective stops. This distinction also exists for affricates, and the syllable structure of the language allows for quite heavy syllables (at least CVCC). There are seven vowel phonemes, and at least nineteen consonant phonemes.

\subsection{Consonant Phonemes}

Table 1 shows the consonant phonemes in Amharic.

\begin{table}[h!]
	\centering
	\caption{Consonant Phonemes}
	\label{consonants}
	\begin{tabular}{|l|l|l|l|l|l|l|l|} \hline
		& Bilabial & Labiodental & Alveolar  & Postalveolar & Palatal & Velar     & Glottal \\ \hline
		Plosive                                                         & p   b    &             & t   t'  d &              &         & k   k'  g & \textipa{P}       \\ \hline
		Nasal                                                           & \hspace{2mm} m        &             & \hspace{2mm} n         &              &         &           &         \\ \hline
		Trill                                                           &          &             & \hspace{2mm} r         &              &         &           &         \\ \hline
		Fricative                                                       & f        &             & s   z     & \textipa{S}           &         &           & h       \\ \hline
		Approximant                                                     &          &             &           &              & \hspace{2mm} j       &           &         \\ \hline
		Lat. approx. &          &             & \hspace{2mm} l         &              &         &           &        \\  \hline
		
	\end{tabular}
Other sounds: w (labio-velar approximant) \\
Affricates: \textteshlig, \textteshlig', ts', \textdyoghlig
\end{table}

There is a two-way contrast for labial stops (voiced and voiceless). Both sounds occur word-initially, as can be seen in (1).
\begin{exe}
	\ex \ipa{pAsti} `pastry'
	\\ \ipa{bAdo} `empty'
\end{exe}

\noindent Alveolar and velar stops have a three-way contrast between voiced, voiceless, and ejective variants. These can be seen in (2) and (3).

\begin{exe}
	\ex \ipa{tAko} `high heel'
	\\ \ipa{t'AbjA} `station'
	\\ \ipa{dAr} `edge'
\end{exe}

\begin{exe}
	\ex \ipa{k1brit} `lighter'
	\\ \ipa{k'odA} `skin'
	\\ \ipa{gAbjA} `blanket'
\end{exe}

\noindent The glottal stop, was not observed word-initially, but it does occur word-medially. It is not clear whether it is a phoneme in the language; there are no close minimal pairs, and it is often deleted.

\begin{exe}
	\ex \ipa{gAPAt} `large bite'
   \\ \ipa{gA:t} `large bite (said quickly'
\end{exe}

Amharic has at least two nasal consonant phonemes. \ipa{m} and \ipa{n} occur word-initially, as in (5). The palatal nasal only occurs word-medially, and so it may be an allophone of [n]. 
\begin{exe}
	\ex \ipa{m1lAs} `tongue'
	\\ \ipa{n1fAs} `wind'
	\ex \ipa{AmAri\textltailn A} `Amharic'
\end{exe}

The trill \ipa{r} has a good minimal pair with \ipa{g}, but there are no word-initial examples in the data.
\begin{exe}
	\ex \ipa{bEr} `door'
	\\  \ipa{bEg} `lamb'
\end{exe}

There are five fricative phonemes in Amharic. All of these occur word-initially, as seen in (8).
\begin{exe}
	\ex \ipa{fErEs} `horse'
	\\ \ipa{sAmbA} `lung'
	\\ \ipa{z1mb} `fly'
	\\ \ipa{SukA} `fork'
	\\ \ipa{hod} `belly'
\end{exe}

There are four affricates in the language, including a voiced, voiceless, and ejective variant of \ipa{\textteshlig}. There were no word-initial examples of the non-ejective \ipa{\textteshlig}, so it may not be a phoneme in the language.

\begin{exe}
	\ex \ipa{\textteshlig'ErEk'A} `moon'
	\ex \ipa{ts'EhAj} `sun'
	\ex \ipa{\textdyoghlig oro} `ear'
	\ex \ipa{Aino\textteshlig u} `eyes'
\end{exe}

Approzimants in Amharic include \ipa{l}, \ipa{j}, and \ipa{w}.
\begin{exe}
	\ex \ipa{lElit} `night'
	\\ \ipa{jimAr1} `bless you'
	\\ \ipa{wEf} `bird'
\end{exe}

\subsection{Vowel Phonemes}

Table 2 shows the vowel phonemes in Amharic.

\begin{table}[h!]
	\centering
	\caption{Vowel Phonemes}
	\label{vowels}
\begin{tabular}{|l|l|l|} \hline
	i     & \textipa{1} & u \\ \hline
	e   \textipa{E} &   & o \\ \hline
	&   & \textipa{A} \\ \hline
\end{tabular} \\
Diphthongs: ej, aj, \textipa{Ew}
\end{table}

The front and mid high vowels \ipa{i} and \ipa{1} occur word-initially and word-finally in (15) and (16), but the back vowel \ipa{u} does not occur word-initially (and is shown word-finally in (15). However, the morpheme /-u/ functions like a determiner, so \ipa{u} we have reason to suspect that it is a phoneme.

\begin{exe}
	\ex \ipa{iju} `look at it'
	\\ \ipa{pAsti} `pastry'
\end{exe}

\begin{exe}
	\ex \ipa{1bet} `house'
	\\ \ipa{jimAr1} `bless you'
\end{exe}

Mid vowels in Amharic include \ipa{e}, \ipa{E}, and \ipa{o}. All occur word-initially and word-finally, as seen in (17), (18), and (19).

\begin{exe}
	\ex \ipa{eli} `turtle'
	\\ \ipa{Ajne} `my eye'
\end{exe}

\begin{exe}
	\ex \ipa{ErE} `really?'
\end{exe}

\begin{exe}
	\ex \ipa{oromi\textltailn a} `language of Oromo people'
	\\ \ipa{doro} `chicken'
\end{exe}

The only low vowel is \ipa{A}, which occurs word-initially and word-finally.
\begin{exe}
	\ex \ipa{AsA} `fish'
\end{exe}

Amharic has three diphthongs. \ipa{ej} only occurs once in the data set, so it may not be a separate phoneme. \ipa{Ew} does not occur word-initially, so it may not be In some cases, these may be better analyzed as a verb and a consonant.
\begin{exe}
	\ex \ipa{ajt} `mouse'
	\\ \ipa{k'ej} `red'
	\\ \ipa{jEsA\textteshlig Ew} `their (formal) hand'
\end{exe}

\subsection{Syllables}

There are several different syllable types in Amharic.

\begin{exe}
	\ex (V) \ipa{eli} `turtle'
	\ex (CV) \ipa{m1lAs} `tongue'
	\ex (CVC) \ipa{wEf} `bird'
	\ex (CVCC) \ipa{wErk'} `gold'
	\ex (CCV) \ipa{Ar1NgwAde} `green'
\end{exe}

In general, we see CCV syllables only with glides as the second syllable. As in (25), (CVCC) syllables are also possible. So, we may propose a general template for syllable structure that looks like (C)(G)V(C)(C), where G is a glide. However, we see no CGVCC syllables, so there may be a limit to how many consonants are in a syllable.

\end{document}

\begin{exe}
	\ex \gll \it \underline{ki}-w-il\\
	3PL.OBJ-1SG.SBJ-see\\
	\trans `I see \underline{them}'
\end{exe}

\begin{table}[h!]
	\centering
	\begin{tabular}{ll}
		\it qo- & us (first person plural) \\
		\it ti- & you (second person plural) \\
		\it ki- & them (third person plural)
	\end{tabular}
	\caption{Object prefixes}
	\label{table:1}
\end{table}



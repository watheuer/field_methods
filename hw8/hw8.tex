\documentclass[12pt]{article}

% packages
\usepackage{tipa}
\usepackage{enumerate}
\usepackage{linguex}
\usepackage{gb4e}
\noautomath
\usepackage{xstring}
\usepackage{multirow}
\usepackage{multicol}
\usepackage{footnote}
\usepackage{qtree}
\usepackage{tree-dvips}
\makesavenoteenv{tabular}
\makesavenoteenv{table}

% define \phon for phonetic examples
\newcommand{\phon}[1]{$[$\textipa{#1}$]$}
\newcommand{\orth}[1]{\textit{\StrSubstitute{#1}{I}{\'{i}}[\x]\StrSubstitute{\x}{E}{\'{e}}[\x]\StrSubstitute{\x}{N}{\~{n}}[\x]\x}}

% margins and spacing
\usepackage[letterpaper, margin=1in]{geometry}
% =================================================================================================

\begin{document}

\begin{center}
{\Large Homework 8} \\
{\large Will Theuer}
\end{center}

\iffalse
Please describe complement clauses in Amharic. By complement clause, I mean sentences that function as arguments (subjects, objects, etc.). We've so far seen several types (patterns using different constructions):

“Imet’alew” ale
'He said, “I am coming”'
 
Indemímet’a belonyal
'He said that he’s coming'

 “K’onjo nat?” t’eyekuwin.
'“Is she pretty?” I asked.'

K’onjo kehonech t’eyekuwin.'
'I asked if she was pretty.' 
 

Meblat ifeligalew.
'I want to eat.'

I would probably have a separate paragraph on each pattern.

Grading rubric. 1. Did you describe the morphological patterns in complement clauses, types of complement clauses, etc.? 2. Were you explicit about the structures (trees) you're assuming? 3. Did you gloss examples, divide them into morphemes, use tabs between words, etc.? 4. Did you give enough examples so that reader would know how to construct examples?
\fi
\section{Complement clauses}

There are several types of complement clauses in Amharic.

\subsection{Null complementizer}

In some cases, complement clauses are formed without an explicit complementizer. This can be seen in (\ref{ex:comp:null1}) and (\ref{ex:comp:null2}).
\begin{exe}
  \ex\label{ex:comp:null1} \gll \orth{i-met'-ale-w} \orth{belo-$\emptyset$-ny-al} \\
  \textsc{g1.pres}-come-\textsc{pres-1s} said-\textsc{3sm.subj}-\textsc{1s.obj}-al \\
  \trans `He said, ``I am coming'''\footnote{The meaning of this -al suffix is unknown.}

  \ex\label{ex:comp:null2} \gll \orth{k'onjo} \orth{nat} \orth{t'eye-kuwin} \\
  pretty be\textbackslash\textsc{3sf} asked-\textsc{1s} \\
  ```Is she pretty?'' I asked'
\end{exe}

\noindent In these sentences, the CP is the object of the main verb. These sentences have the following structure:

\begin{exe}
\ex \Tree [.S [.NP N\\$\emptyset$ ] [.VP [.CP [.S [.NP N\\$\emptyset$ ] [.VP V\\\orth{imet'alew} ] ] C\\$\emptyset$ ] V\\\orth{belonyal} ] ]
\end{exe}


\subsection{Complementizer prefixes}

There are several examples of verbal prefixes that function as complementizers. In many of these examples,  \orth{inde-} is used; this is shown in (\ref{ex:comp:inde1}) and (\ref{ex:comp:inde2}). 
\begin{exe}
  \ex\label{ex:comp:inde1} \gll \orth{inde-mI-met'a} \orth{belo-$\emptyset$-ny-al} \\
  \textsc{comp}-\textsc{3sm.comp}-come\textbackslash\textsc{3sm.past} said-\textsc{3sm.subj}-\textsc{1s.obj}-al \\
  \trans `He said that he's coming'

  \ex\label{ex:comp:inde2} \gll \orth{temarI} \orth{inde-hon-ku} \orth{tenager-ku} \\
  student \textsc{comp}-be-\textsc{1s.subj} said-\textsc{1s.subj} \\
  \trans `I said that I am a student'
\end{exe}

\noindent Another such prefix is \orth{ke-}, which functions similarly in (\ref{ex:comp:ke}). 

\begin{exe}
  \ex\label{ex:comp:ke} \gll \orth{k'onjo} \orth{ke-hon-ech} \orth{t'eye-kuwin} \\
  pretty \textsc{comp}-be-\textsc{3sf} asked-\textsc{1s} \\
  \trans `I asked if she was pretty'
\end{exe}

In some examples, these prefixes are accompanied by a variation of \orth{-mi-} and even a verbal suffix in the \textsc{1sf}, \textsc{2p}, and \textsc{3p} forms. It is unclear when these are used, as they occur in forms like (\ref{ex:comp:inde1}) but not in forms like (\ref{ex:comp:inde2}). This paradigm is shown in Table \ref{tab:comp:forms}.

\begin{table}[h!t]
\centering
\caption{Complementizer forms}
\label{tab:comp:forms}
\begin{tabular}{lll}
  \textsc{1s} & \orth{meblat inde-mi-felig} & `that I want to eat' \\
  \textsc{2sm} & \orth{meblat inde-mit-felig} & `that you (m) want to eat' \\
  \textsc{2sf} & \orth{meblat inde-mit-felig-I} & `that you (f) want to eat' \\
  \textsc{3sm} & \orth{meblat inde-mI-felig} & `that he wants to eat' \\
  \textsc{3sf} & \orth{meblat inde-mit-felig} & `that she wants to eat' \\
  \textsc{1p} & \orth{meblat inde-mini-felig} & `that we want to eat' \\
  \textsc{2p} & \orth{meblat inde-miti-felig-u} & `that you (pl) want to eat' \\
  \textsc{3p} & \orth{meblat inde-felig-u} & `that they want to eat' 
\end{tabular}
\end{table}

(\ref{ex:comp:inde:tree}) shows the structure of these sentences; the second tree shows how the verb moves up to the complementizer \orth{inde-}. As in the previous examples, the CP is the object of the verb \orth{belonyal}.

\begin{exe}
  \ex\label{ex:comp:inde:tree} \Tree [.S [.NP N\\$\emptyset$ ] [.VP [.CP [.S [.NP N\\$\emptyset$ ] [.VP V\\\orth{mImet'a} ] ] C\\\orth{inde-} ] V\\\orth{belonyal} ] ]
\Tree [.S [.NP N\\$\emptyset$ ] [.VP [.CP [.S [.NP N\\$\emptyset$ ] [.VP V\\$t$ ] ] C\\\orth{inde-mImet'a} ] V\\\orth{belonyal} ] ]
\end{exe}

\subsection{Infinitives}

Infinitives are formed with the verbal prefix \orth{me-} and function like other prefix complementizers. This is shown in (\ref{ex:inf1}), (\ref{ex:inf2}), and (\ref{ex:inf3}).

\begin{exe}
  \ex\label{ex:inf1} \gll \orth{me-blat} \orth{i-felig-ale-w} \\
  \textsc{inf}-eat \textsc{g1.pres}-want-\textsc{pres-1s} \\
  \trans `I want to eat'

  \ex\label{ex:inf2} \gll \orth{me-blat} \orth{inde-mIfelig} \orth{negro-$\emptyset$-nya} \\
  \textsc{inf}-eat \textsc{comp}-want told-\textsc{3sm.subj-1s.obj} \\
  \trans `He told me that he wants to eat'

  \ex\label{ex:inf3} \gll \orth{ruz} \orth{me-blat} \orth{ti-felig-achu} \orth{wey?} \\
  rice \textsc{inf}-eat \textsc{g2.pres}-want-\textsc{2p} right? \\
  `You all want to eat rice, right?'
\end{exe}

The structure of these sentences is shown in (\ref{ex:inf:tree}). The verb \orth{blat} `eat' moves up from the V to the C (\orth{me-}).

\begin{exe}
  \ex\label{ex:inf:tree} \Tree [.S [.NP N\\$\emptyset$ ] [.VP [.CP [.S [.NP N\\$\emptyset$ ] [.VP [.NP N\\\orth{ruz} ] V\\$t$ ] ] C\\\orth{me-blat} ] V\\\orth{tifeligachu} ] ]
\end{exe}

\end{document}

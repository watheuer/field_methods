\documentclass[12pt]{article}

% packages
\usepackage{tipa}
\usepackage{enumerate}
\usepackage{linguex}
\usepackage{gb4e}
\noautomath
\usepackage{xstring}
\usepackage{multirow}
\usepackage{footnote}
\makesavenoteenv{tabular}
\makesavenoteenv{table}

% define \phon for phonetic examples
\newcommand{\phon}[1]{$[$\textipa{#1}$]$}
\newcommand{\orth}[1]{\textit{\StrSubstitute{#1}{I}{\'{i}}[\x]\StrSubstitute{\x}{E}{\'{e}}[\x]\StrSubstitute{\x}{N}{\~{n}}[\x]\x}}

% margins and spacing
\usepackage[letterpaper, margin=1in]{geometry}

\iffalse
Grading rubric: Is your segmentation of words into morphemes sensible and consistent? Did you describe all the variants of the affixes? Is your description well-organized with clear paragraphs and transitions between examples and tables? Is your format appropriate and your terms clearly defined? Is your writing carefully edited?
\fi


\begin{document}

\begin{center}
{\Large Homework 6} \\
{\large Will Theuer}
\end{center}

\section{Subject person marking}

In Amharic, subject person marking differs in each tense.

\subsection{Past positive, negative}
\iffalse
*Past tense in positive and negative.* Please describe the person marking in intransitive verbs in the past tense in the positive and negative. According to my notes, we have these forms for merot' 'to run', mezemir 'to sing', mesrat 'to work', and meblat 'to eat'. You might use a layout like this:

(1)   sera-w 'I worked'                al-sera-w-m 'I didn't work'

        sera-h 'you m. worked'      al-sera-h-im 'you m. didn't work'
\fi

In the past tense, there are different affixes depending on whether the verb root ends in a consonant or a vowel. Example (\ref{ex:past:sing}) lists the different forms for a verb ending in a consonant.

\begin{exe}
  \ex\label{ex:past:sing} \orth{mezemir} `to sing' \\
  \begin{tabular}{ll}
    \orth{zemer-ku}\footnote{Arsima was unsure if this form was correct, and also provided \orth{zemerkuwin}} `I sang' & \orth{al-zemer-ku-m} `I didn't sing' \\
    \orth{zemer-k} `You (m) sang' & \orth{al-zemer-k-im} `You (m) didn't sing \\
    \orth{zemer-sh} `You (f) sang' & \orth{al-zemer-sh-im} `You (f) didn't sing \\
    \orth{zemer-e} `He sang' & \orth{al-zemer-e-m} `He didn't sing' \\
    \orth{zemer-ech} `She sang' & \orth{al-zemer-ech-im} `She didn't sing' \\
    \orth{zemer-en} `We sang' & \orth{al-zemer-en-im} `We didn't sing' \\
    \orth{zemer-achu} `You (pl) sang' & \orth{al-zemer-achu-m} `You (pl) didn't sing' \\
    \orth{zemer-u} `They sang' & \orth{al-zemer-u-m} `They didn't sing' \\
  \end{tabular}
\end{exe}

\noindent By contrast, (\ref{ex:past:work}) shows the forms for a verb root ending in a consonant.

\begin{exe}
  \ex\label{ex:past:work} \orth{mesrat} `to work' \\
  \begin{tabular}{ll}
    \orth{sera-w} `I worked' & \orth{al-sera-w-m} `I didn't work' \\
    \orth{sera-h} `You (m) worked' & \orth{al-sera-h-im} `You (m) didn't work \\
    \orth{sera-sh} `You (f) worked' & \orth{al-sera-sh-im} `You (f) didn't work \\
    \orth{sera} `He worked' & \orth{al-sera-m} `He didn't work' \\
    \orth{sera-ch} `She worked' & \orth{al-sera-ch-im} `She didn't work' \\
    \orth{sera-n} `We worked' & \orth{al-sera-n-im} `We didn't work' \\
    \orth{sera-chu} `You (pl) worked' & \orth{al-sera-chu-m} `You (pl) didn't work' \\
    \orth{ser-u} `They worked' & \orth{al-ser-u-m} `They didn't work' \\
  \end{tabular}
\end{exe}

All V-final forms preserve the final root vowel with the exception of the third-person plural form \orth{seru}. In this form, the final vowel is dropped and replaced with the \orth{-u} suffix; otherwise, it would be indistinguishable from the \textsc{3sm} form. These two paradigms are summarized in table \ref{tab:past:subject-affixes}.

\begin{table}[h!t]
\centering
\caption{Past subject affixes}
\label{tab:past:subject-affixes}
  \begin{tabular}{l|ll}
    form & C-final & V-final \\ \hline
    \textsc{1s}  & \orth{-ku}   & \orth{-w} \\ 
    \textsc{2sm} & \orth{-k}    & \orth{-h} \\
    \textsc{2sf} & \orth{-sh}   & \orth{-sh} \\
    \textsc{3sm} & \orth{-e}    & \orth{-$\emptyset$} \\
    \textsc{3sf} & \orth{-ech}  & \orth{-ch} \\
    \textsc{1p}  & \orth{-en}   & \orth{-n} \\
    \textsc{2p}  & \orth{-achu} & \orth{-chu} \\
    \textsc{3p}  & \orth{-u}    & \orth{-u}
  \end{tabular}
\end{table}

\newpage
\subsection{Present/future positive, negative}
\iffalse
*Present/future tense in positive and negative.* Please describe the person marking in intransitive verbs in the present/future tense in the positive and negative. I think we only have these forms for merot' 'to run', collected on 2/16 (irot'alew 'I run', alrot'im 'I don't run'). These forms are surprising and quite challenging.
\fi

In the present tense, there are several different forms for the positive and negative. Each form requires a prefix and a suffix, which is summarized in table \ref{tab:pres:affixes}.

\begin{exe}
  \ex\label{ex:pres:run} \orth{merot'} `to run' \\
  \begin{tabular}{ll}
    \orth{inE i-rot'-al-ew} `I run'              & \orth{inE al-rot'-im} `I don't run' \\
    \orth{ante ti-rot'-al-eh} `You (m) run'      & \orth{ante at-rot'-im} `You (m) don't run \\
    \orth{anchI ti-roch'-al-esh} `You (f) run'   & \orth{anchI at-roch'-im} `You (f) don't run \\
    \orth{isu yI-rot'-al-e} `He run'             & \orth{isu ay-rot'-im} `He doesn't run' \\
    \orth{iswa ti-rot'-al-ech} `She run'         & \orth{iswa at-rot'-im} `She doesn't run' \\
    \orth{inya in-rot'-al-en} `We run'           & \orth{inya an-rot'-im} `We don't run' \\
    \orth{inante ti-rot'-al-achu} `You (pl) run' & \orth{inante at-rot'-um} `You (pl) don't run' \\
    \orth{inesu yI-rot'-al-u} `They run'         & \orth{inesu ay-rot'-um} `They don't run' \\
  \end{tabular}
\end{exe}

\begin{table}[ht]
\centering
\caption{Present/future affixes}
\label{tab:pres:affixes}
  \begin{tabular}{l|lll}
    form & positive & negative & group \\ \hline
    \textsc{1s}  & \orth{i- -al-ew}    & \orth{al- -im} & 1 (\orth{i-}/\orth{al-})\\ 
    \textsc{2sm} & \orth{ti- -al-eh}   & \orth{at- -im} & 2 (\orth{ti-}/\orth{at-})\\
    \textsc{2sf} & \orth{ti- -al-esh}  & \orth{at- -im} + palatalization & 2 (\orth{ti-}/\orth{at-})\\
    \textsc{3sm} & \orth{yI- -al-e}    & \orth{ay- -im} & 3 (\orth{yI-}/\orth{ay-})\\
    \textsc{3sf} & \orth{ti- -al-ech}  & \orth{at- -im} & 2 (\orth{ti-}/\orth{at-})\\
    \textsc{1p}  & \orth{in- -al-en}   & \orth{an- -im} & 4 (\orth{in-}/\orth{an-})\\
    \textsc{2p}  & \orth{ti- -al-achu} & \orth{at- -um} & 2 (\orth{ti-}/\orth{at-})\\
    \textsc{3p}  & \orth{yI- -al-u}    & \orth{ay- -um} & 3 (\orth{yI-}/\orth{ay-}) 
  \end{tabular}
\end{table}

\section{Object person marking in transitive verbs}
\iffalse
*Object person marking.* Please describe the object person marking in transitive verbs in the past tense (i.e., for verbs that mark the person of the subject and object). You should give a table of the suffixes and then verb forms that provide evidence for the table. You don't need to give 8 x 8 verb forms: just enough to demonstrate the object person markers and at least some subject markers to show how they interact.

Please be sure to mention what "object" means: does the verb agree with the direct object, the indirect object, both, or something else?
\fi

In transitive verbs, the direct object is marked by a verbal suffix which occurs after the subject suffix (listed in table \ref{tab:pres:affixes}).

\begin{exe}
  \ex \gll \orth{ak'if-ke-ny} \\
  hug-\textsc{2sm.subj}-\textsc{1s.obj} \\
  \trans `You (m) hugged me'
\end{exe}

\begin{table}[ht]
\centering
\caption{Subject and object suffixes}
\label{tab:person-marking}
  \begin{tabular}{l|ll}
    form & subject &  object \\ \hline
    \textsc{1s}  & \orth{-ku}    & \orth{-eny} \\ 
    \textsc{2sm} & \orth{-k}    & \orth{-eh} \\
    \textsc{2sf} & \orth{-sh}   & \orth{-ish} \\
    \textsc{3sm} & \orth{-e}     & \orth{-t, -ew} \\
    \textsc{3sf} & \orth{-ech}  & \orth{-(w)at} \\
    \textsc{1p}  & \orth{-en}    & \orth{-en}\footnote{This possibly becomes \orth{-n} after a palatalized consonant.} \\
    \textsc{2p}  & \orth{-achu-} & \orth{-(w)achu} \\
    \textsc{3p}  & \orth{-u}     & \orth{-(w)achew}
  \end{tabular}
\end{table}

\begin{exe}
  \ex \orth{mak'if} `to hug' \\
  \begin{tabular}{ll}
    \orth{ak'if-e-ny}  & `He hugged me' \\
    \orth{ak'if-e-h}   & `He hugged you (m)' \\
    \orth{ak'if-e-sh}  & `He hugged you (f)' \\
    \orth{ak'if-e-w}   & `He hugged him' \\
    \orth{ak'if-at}    & `He hugged her' \\
    \orth{ak'if-e-n}   & `He hugged us' \\
    \orth{ak'if-achu}  & `He hugged you (pl)' \\
    \orth{ak'if-achew} & `He hugged them' 
  \end{tabular}
\end{exe}

\end{document}

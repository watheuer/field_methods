\documentclass[12pt]{article}

% packages
\usepackage{tipa}
\usepackage{enumerate}
\usepackage{linguex}
\usepackage{gb4e}
\noautomath
\usepackage{xstring}
\usepackage{multirow}
\usepackage{multicol}
\usepackage{footnote}
\usepackage{qtree}
\usepackage{tree-dvips}
\makesavenoteenv{tabular}
\makesavenoteenv{table}

% define \phon for phonetic examples
\newcommand{\ipa}[1]{$/$\textipa{#1}$/$}
\newcommand{\phon}[1]{$[$\textipa{#1}$]$}
\newcommand{\orth}[1]{\textit{\StrSubstitute{#1}{I}{\'{i}}[\x]\StrSubstitute{\x}{E}{\'{e}}[\x]\StrSubstitute{\x}{N}{\~{n}}[\x]\x}}

% margins and spacing
\usepackage[letterpaper, margin=1in]{geometry}
% =================================================================================================

\begin{document}

\begin{center}
{\Large Amharic} \\
{\large Will Theuer}
\end{center}

\section{Overview}
\setcounter{exx}{0}

\section{Phonology}
\setcounter{exx}{0}

Amharic has several interesting phonological features, including a three-way contrast for stops: voiced, voiceless, and ejective. This distinction also exists for affricates, and the syllable structure of the language allows for quite heavy syllables (at least CVCC). There are seven vowel phonemes, and at least nineteen consonant phonemes.

\subsection{Consonant Phonemes}

Table 1 shows the consonant phonemes in Amharic.

\begin{table}[ht!]
\centering
\caption{Consonant Phonemes}
\label{tab:consonants_ipa}
\begin{tabular}{|l|l|l|l|l|l|l|l|} \hline
             & Bilabial & Labiodent. & Alveolar  & Postalveolar & Palatal & Velar     & Glottal \\ \hline
Plosive      & p   b    &             & t   t'  d &              &         & k   k'  g & \textipa{P}       \\ \hline
Nasal        & \hspace{2mm} m        &             & \hspace{2mm} n         &              &         &           &         \\ \hline
Trill        &          &             & \hspace{2mm} r         &              &         &           &         \\ \hline
Fricative    &          & f           & s   z     & \textipa{S}           &         &           & h       \\ \hline
Approximant  &          &             &           &              & \hspace{2mm} j       &           &         \\ \hline
Lat. approx. &          &             & \hspace{2mm} l         &              &         &           &        \\  \hline
	
\end{tabular}
Other sounds: w (labio-velar approximant) \\
Affricates: \textteshlig, \textteshlig', ts', \textdyoghlig
\end{table}

% TODO: talk about p'?
The plosives in Aharic are \ipa{p b t t' d k k' g P}. There is a two-way contrast for labial stops (voiced and voiceless). Both sounds occur word-initially, as can be seen in (1).
\begin{exe}
	\ex \ipa{pAsti} `pastry'
	\\ \ipa{bAdo} `empty'
\end{exe}

\noindent Alveolar and velar stops have a three-way contrast between voiced, voiceless, and ejective variants. These can be seen in (2) and (3).

\begin{multicols}{2}
\begin{exe}
  \ex \ipa{tAko} `high heel'
  \\ \ipa{t'AbjA} `station'
  \\ \ipa{dAr} `edge'
  \ex \ipa{k1brit} `lighter'
  \\ \ipa{k'odA} `skin'
  \\ \ipa{gAbjA} `blanket'
\end{exe}
\end{multicols}

\noindent The glottal stop was not observed word-initially, but it does occur word-medially. It is not clear whether it is a phoneme in the language; there are no close minimal pairs, and it is often deleted.

\begin{exe}
  \ex \ipa{gAPAt} `large bite'
   \\ \ipa{gA:t} `large bite (said quickly'
\end{exe}

Amharic has at least two nasal consonant phonemes. \ipa{m} and \ipa{n} occur word-initially, as in (5). The palatal nasal only occurs word-medially, and so it may be an allophone of [n]. 
\begin{exe}
  \ex \ipa{m1lAs} `tongue'
  \\ \ipa{n1fAs} `wind'
  \ex \ipa{AmAri\textltailn A} `Amharic'
\end{exe}
In the rest of the paper, the palatal nasal is represented with $<$nj$>$.

There is one trill in the language, \ipa{r}, which has a good minimal pair with \ipa{g} in (\ref{phon:r_g}). We also see a word-initial example in (\ref{phon:r}).
\begin{exe}
  \ex\label{phon:r_g} \ipa{bEr} `door'
  \\  \ipa{bEg} `lamb'

  \ex\label{phon:r} \ipa{rE\textdyoghlig im} `tall'
\end{exe}

There are five fricative phonemes in Amharic, \ipa{f s z S h}. All of these occur word-initially, as seen in (8).
\begin{exe}
  \ex \ipa{fErEs} `horse'
  \ex \ipa{sAmbA} `lung'
  \ex \ipa{z1mb} `fly'
  \ex \ipa{SukA} `fork'
  \ex \ipa{hod} `belly'
\end{exe}

\noindent Additionally, there are four affricates in the language, including a voiced, voiceless, and ejective variant of \ipa{\textteshlig}. There were no word-initial examples of the non-ejective \ipa{\textteshlig}, so it may not be a phoneme in the language.

\begin{exe}
  \ex \ipa{\textteshlig'ErEk'A} `moon'
  \ex \ipa{ts'EhAj} `sun'
  \ex \ipa{\textdyoghlig oro} `ear'
  \ex \ipa{Aino\textteshlig u} `eyes'
\end{exe}

Approximants in Amharic include \ipa{l}, \ipa{j}, and \ipa{w}, which all occur word-initially.
\begin{exe}
  \ex \ipa{lElit} `night'
  \\ \ipa{jimAr1} `bless you'
  \\ \ipa{wEf} `bird'
\end{exe}
\ipa{w} also occurs as part of the diphthong \ipa{aw}.

\subsection{Vowel Phonemes}

Table 2 shows the vowel phonemes in Amharic.

\begin{table}[ht!]
\centering
\caption{Vowel Phonemes}
\label{vowels}
\begin{tabular}{|l|l|l|} \hline
  i   & \textipa{1} & u \\ \hline
  e   \textipa{E} &   & o \\ \hline
  &   & \textipa{A} \\ \hline
\end{tabular} \\
Diphthongs: ej, aj, \textipa{Ew}, \textipa{Aw}
\end{table}

The front and mid high vowels \ipa{i} and \ipa{1} occur word-initially and word-finally in (15) and (16), but the back vowel \ipa{u} does not occur word-initially (and is shown word-finally in (15). However, the morpheme /-u/ is a definite marker in the language, so \ipa{u} we have reason to suspect that it is a phoneme.

\begin{exe}
  \ex \ipa{iju} `look at it'
  \\ \ipa{pAsti} `pastry'
  \ex \ipa{1bet} `house'
  \\ \ipa{jimAr1} `bless you'
\end{exe}

Mid vowels in Amharic include \ipa{e}, \ipa{E}, and \ipa{o}. All occur word-initially and word-finally, as seen in (17), (18), and (19).

\begin{exe}
  \ex \ipa{eli} `turtle'
  \\ \ipa{Ajne} `my eye'
  \ex \ipa{ErE} `really?'
  \ex \ipa{oromi\textltailn a} `language of Oromo people'
  \\ \ipa{doro} `chicken'
\end{exe}

The only low vowel is \ipa{A}, which occurs word-initially and word-finally.
\begin{exe}
  \ex \ipa{AsA} `fish'
\end{exe}

Amharic has four diphthongs, \ipa{aj ej Ew Aw}. \ipa{Ew} does not occur word-initially, so it may not be a phoneme. In some cases, these may be better analyzed as a vowel and a consonant.
\begin{exe}
  \ex \ipa{ajt} `mouse'
  \ex \ipa{k'ej} `red'
  \ex \ipa{jEsA\textteshlig Ew} `their (formal)'
  \ex \ipa{bElAw} `I ate'
\end{exe}

\subsection{Syllables}

There are several different syllable types in Amharic.

\begin{exe}
  \ex (V) \ipa{eli} `turtle'
  \ex (CV) \ipa{m1lAs} `tongue'
  \ex (CVC) \ipa{wEf} `bird'
  \ex (CVCC) \ipa{wErk'} `gold'
  \ex (CCV) \ipa{Ar1NgwAde} `green'
\end{exe}

In general, we see CCV syllables only with glides as the second consonant. As in (25), (CVCC) syllables are also possible. So, we may propose a general template for syllable structure that looks like (C)(G)V(C)(C), where G is a glide. However, we see no CGVCC syllables, so there may be a limit to how many consonants are in a syllable.

\subsection{Orthography}
The following orthography was created for easier transcription of Amharic and is used in the rest of the paper.

\begin{table}[ht]
\centering
\caption{Vowels}
\label{tab:orthography-vowels}
\begin{tabular}{lll}
  \orth{I} \phon{i} & \orth{i} \phon{1} & \orth{u} \phon{u} \\
  \orth{E} \phon{e} &                   & \orth{o} \phon{o} \\
  \orth{  e} \phon{E} &                   & \orth{a} \phon{A} \\
\end{tabular}
\end{table}

\noindent Consonants remain largely the same as their IPA equivalents, with the exception of the affricates.

\begin{table}[ht]
\centering
\caption{Consonants}
\label{tab:consonants_orthography}
\begin{tabular}{llllllll}
 Bilabial & Labiodental & Alveolar  & Postalveolar & Palatal & Velar     & Glottal \\ \hline
 p   b    &             & t   t'  d &              &         & k   k'  g & '\phon{P}       \\
 m        &             & n         &              &         &           &         \\ 
          &             & r         &              &         &           &         \\ 
          & f           & s   z     & sh\phon{\textipa{S}}  &         &           & h       \\
          &             &           &              & y\phon{j}&           &         \\
          &             & l         &              &         &           &        \\ 
        
\end{tabular}\\
Other sounds: w (labio-velar approximant) \\
Affricates: ch\phon{\textteshlig}, ch'\phon{\textteshlig'}, ts', j\phon{\textdyoghlig}
\end{table}

\newpage
\section{Pronouns}
\setcounter{exx}{0}

\iffalse
Pronouns:
What types of pronouns are there and what names will you give them? So far I see words like iné  'I', ihé 'this', and yené 'mine'. I would probably list interrogative pronouns in a chapter on questions.
Grading rubric: Did you describe all the patterns in the data in a logical, well-ordered way? Do you rise above details to give a good feel for how the language works? Did you include enough data to document the language? Is your data accurate? Are you using normal format and terms for describing a language? Do you define your terms? Is your writing carefully edited?
\fi

Subject pronouns have the following forms in Amharic:

\begin{table}[ht]
\centering
\caption{Subject pronouns}
\label{tab:pronouns}
  \begin{tabular}{ll}
    \orth{inE} & I \\
    \orth{ante} & you (m) \\
    \orth{anchI} & you (f) \\
    \orth{isu} & he \\
    \orth{iswa} & she \\
    \orth{inya} & we \\
    \orth{inante} & you (pl) \\
    \orth{inesu} & they \\
  \end{tabular}
\end{table}

These pronouns only occur as the subjects of sentences, and they occur before objects and verbs. This provides some evidence that suggests that Amharic is an SOV language.

\begin{exe}
  \ex \gll \orth{inE} \orth{rejim} \orth{neny} \\
  I tall be\textbackslash\textsc{1s} \\
  \trans  `I am tall.'

  \ex \gll \orth{ante} \orth{rejim} \orth{neh}\\
  you-\textsc{m} tall be\textbackslash\textsc{2ms}\\
  \trans `You (m) are tall'

  \ex \gll \orth{anchI} \orth{rejim} \orth{nesh}\\
  you-\textsc{f} tall be\textbackslash\textsc{2fs}\\
  \trans `You (f) are tall'

  \ex \gll \orth{isu} \orth{rejim} \orth{new}\\
  he tall be\textbackslash\textsc{3ms}\\
  \trans `He is tall'

  \ex \gll \orth{iswa} \orth{rejim} \orth{nat}\\
  she tall be\textbackslash\textsc{3fs}\\
  \trans `She is tall'

  \ex \gll \orth{inya} \orth{rejim} \orth{nen}\\
  we tall be\textbackslash\textsc{1p}\\
  \trans `We are tall'

  \ex \gll \orth{inant} \orth{rejim} \orth{nachu}\\
  you\textbackslash\textsc{p} tall be\textbackslash\textsc{2p}\\
  \trans `You (pl) are tall'


  \ex \gll \orth{inesu} \orth{rejim} \orth{nachew}\\
  they tall be\textbackslash\textsc{3p}\\
  \trans `They are tall'
\end{exe}


Additionally, the language has demonstrative pronouns (like `this' and `that'). These pronouns appear to be grammatically masculine; in (\ref{ex:this}), we see that the verb \orth{new} `to be' is in the masculine form, even though \orth{wef} `bird' is grammatically feminine. 

\begin{exe}
  \ex\label{ex:this} \gll \orth{ihE} \orth{k'ey} \orth{wef} \orth{new} \\
  this red bird be\textbackslash\textsc{3ms}\\
  \trans `This is a red bird'\footnote{In these glosses, \textsc{3fs} means third person feminine singular, \textsc{3pl} means third person plural, etc.} 

  \ex \gll \orth{inezI} \orth{k'ey} \orth{wef-och} \orth{nachew} \\
  these red bird-\textsc{p} be\textbackslash\textsc{3p}\\
  `These birds are red'

  \ex \gll \orth{yachI} \orth{wef} \orth{tinishIyE} \orth{nat} \\
  that bird small be\textbackslash\textsc{3fs} \\
  `That bird is small'
\end{exe}

There is evidence of possessive pronouns, but we only see a single example in the data.

\begin{exe}
  \ex \orth{yenE} 'mine'
\end{exe}


\newpage
\section{Possession}
\setcounter{exx}{0}

There are two different strategies for possession in Amharic, both of which are used for alienable and inalienable possession.

\subsection{Suffixes}

Possession is marked with noun suffixes. Second and third person singular suffixes distinguish between male and femaile, but the plural forms are not gendered.

\begin{exe}
  \ex \orth{doro-E} `my chicken'\\
      \orth{doro-i} `your (m) chicken'\\
      \orth{doro-ish} `your (f) chicken'\\
      \orth{doro-u} `his chicken'\\
      \orth{doro-wa} `her chicken'
\end{exe}

\noindent The second person masculine suffix also occurs as \orth{-ih}.
\begin{exe}
  \ex\label{ex:male_suffix_h} \orth{ij-ih} `your (m) hand'
\end{exe}

All plural suffixes begin with \orth{ach-}. These suffixes may be further analyzable as \orth{-ach-in, -ach-un, and -ach-ew}.
\begin{exe}
  \ex \orth{doro-achin} `our chicken' \\
      \orth{doro-achun} `your (pl) chicken'\\
      \orth{doro-achew} `their chicken'
\end{exe}

\subsection{Contrast forms}

Additionally, Amharic has several contrast forms for possession. These forms place the stress on the possessor and convey a meaning like `MY hand.' Each of these can also be expressed with a suffix form as in the previous section.

\begin{table}[ht]
\centering
\caption{Contrast forms}
\label{tab:contrast}
  \begin{tabular}{llll}
    \orth{yenE ij}      & `my hand'             & \orth{ij-E} & `my hand' \\
    \orth{yante ij}     & `your (m) hand'       & \orth{ij-ih} & `your (m) hand'\\
    \orth{yanchI ij}    & `your (f) hand'       & \orth{ij-ish} & `your (f) hand'\\
    \orth{yesu ij}      & `his hand'            & \orth{ij-u} & `his hand'\\
    \orth{yeswa ij}     & `her hand'            & \orth{ij-wa} & `her hand'\\
    \orth{yenya ij-och} & `our hands'           & \orth{ij-och-achin} & `our hands'\\
    -\footnote{There was no example of a contrast form for the second person plural (you all).} & - & \orth{ij-och-achun} & `your (pl) hands'  \\ 
    \orth{yenesu ij-och} & `their hands' & \orth{ij-och-achew} & `their hands' \\
 \end{tabular}
\end{table}

Possessive forms like `Rachel's hand' are constructed similarly. As shown in (\ref{ex:bad_possession}), these constructions do not work with the suffix forms in the previous section.

\begin{exe}
  \ex\label{ex:possession_female} \orth{je Rachel ij} `Rachel's hand'
  \ex\label{ex:bad_possession} *\orth{Rachel ijwa} `Rachel's hand'
  \ex\label{ex:possession_male} \orth{je Abe ij} `Abe's hand'
\end{exe}

\noindent We can see in (\ref{ex:possession_female}) and (\ref{ex:possession_male}) that these forms are the same for male and female.

\newpage
\section{Reflexives}
\setcounter{exx}{0}

Reflexives are formed with the noun \orth{iras}, which may be derived from \orth{ras} `head.' \orth{iras} is the object of the sentence and takes a possessive and a \orth{-n/-ny} object suffix. In (\ref{ex:reflexive:wash}), we see the paradigm for the verb \orth{metat'eb} `to wash.'

\begin{exe}
  \ex\label{ex:reflexive:wash} \orth{metat'eb} `to wash' \\
  \begin{tabular}{lll}
    \orth{iras-E-ny} & \orth{tat'eb-kuwin} & `I washed myself' \\
    \orth{iras-i-n} & \orth{tat'eb-k} & `You (m) washed yourself'\\
    \orth{iras-ish-in} & \orth{tat'eb-sh} & `You (f) washed yourself'\\
    \orth{iras-u-n} & \orth{tat'eb-e} & `He washed himself'\\
    \orth{iras-wa-n} & \orth{tat'eb-ech} & `She washed herself'\\
    \orth{iras-achin} & \orth{tat'eb-en} & `We washed ourselves'\\
    \orth{iras-achun} & \orth{tat'eb-achu} & `You all washed yourselves'\\
    \orth{iras-achew-n} & \orth{tat'eb-u} & `They washed themselves'
  \end{tabular}
\end{exe}

Each of these forms of \orth{iras} has an added \orth{-n/-ny/-in} object suffix unless the possessive suffix already ends in \orth{n}\footnote{This is probably an object suffix that occurs throughout the language. This requires some more data to prove.}. The \orth{-ny} form occurs after the tense \orth{-E} suffix in the \textsc{1s} form, and the \orth{-in} form occurs after \orth{-ish}. This may be a result of the cluster \orth{*ishn} not being allowed in the language. (\ref{ex:reflexive:wash:gll1}) and (\ref{ex:reflexive:wash:gll2}) show the glosses of two of the forms.

\begin{exe}
  \ex\label{ex:reflexive:wash:gll1} \gll \orth{iras-E-ny} \orth{tat'eb-kuwin} \\
  \textsc{refl}-\textsc{1s.poss}-\textsc{obj} wash-\textsc{1s} \\
  \trans `I washed myself'

  \ex\label{ex:reflexive:wash:gll2} \gll \orth{iras-i-n} \orth{tat'eb-k} \\
  \textsc{refl}-\textsc{2sm.poss}-\textsc{obj} wash-\textsc{2sm} \\
  \trans `You (m) washed yourself'
\end{exe}

The verb form is conjugated as we expect from section ??\footnote{This will point to the actual section in the final write-up...} on subject person marking. It agrees with the subject of the sentence, which either occurs before the object or only as a verbal suffix (as seen here).


\newpage
\section{Commands}
\setcounter{exx}{0}
There are four different command forms in Amharic for male, female, and plural recipients in addition to the hortative form. These forms agree with the recepient using a verbal suffix in the male, female, and plural forms, and a prefix in the hortative form.

%\begin{table}[ht]
%\centering
%\caption{Command forms}
%\label{tab:commands-data}
%\begin{tabular}{l|lll|llll}
%     & \multicolumn{3}{c}{group 1}                                          & \multicolumn{4}{c}{group 2} \\
%     & \orth{meblat} & \orth{metENat} & \orth{met'et'at}  & \orth{mambib}   & \orth{merot'}    & \orth{mets'af}   & \orth{mehEd} \\ \hline
%  F  & \orth{bI}     & \orth{tENI}    & \orth{t'ech'I}    & \orth{ambibI}   & \orth{ruch'I}    & \orth{ts'afI}    & \orth{hIjI} \\
%  M  & \orth{bila}   & \orth{tENa}    & \orth{t'et'a}     & \orth{ambib}    & \orth{rut'}      & \orth{ts'af}     & \orth{hId} \\
%  P  & \orth{bilu}   & \orth{tENu}    & \orth{t'et'u}     & \orth{ambibu}   & \orth{rut'u}     & \orth{ts'afu}    & \orth{hIdu} \\
%  H  & \orth{inibla} & \orth{initENa} & \orth{init'eta}   & \orth{inambib}  & \orth{inirut'}   & \orth{inits'af}  & \orth{inihId} \\
%  NF & \orth{atibI}  & \orth{atitENI} & \orth{atit'ech'I} & \orth{atambibI} & \orth{atiruch'I} & \orth{atits'afI} & \orth{atihIjI} \\
%  NM & \orth{atibla} & \orth{atitENa} & \orth{atit'et'a}  & \orth{atambib}  & \orth{atirut'}   & \orth{atits'af}  & \orth{atihId} \\
%  NP & \orth{atiblu} & \orth{atitENu} & \orth{atit'et'u}  & \orth{atambibu} & \orth{atirut'u}  & \orth{atits'afu} & \orth{atihIdu} \\
%  NH & \orth{anibla} & \orth{anitENa} & \orth{anit'et'a}  & \orth{anambib}  & \orth{anirut'}   & \orth{anits'af}  & \orth{anihId} \\
%\end{tabular}
%\end{table}


%\noindent The language may require verbs to end in a consonant in their infinitive forms. Therefore, verbs in group 1 take an extra /-t/, while the verbs in group 2 already have a final consonant as part of the verb root. Alternatively, the /-t/ suffix may exist underlyingly for every infinitive, and the word-final custers that would be produced in group 2 verbs are not allowed. The verbs from table \ref{tab:commands-data} have the following roots:
%
%\begin{table}[ht]
%\centering
%\caption{Verb roots}
%\label{tab:commands-roots}
%\begin{tabular}{l|lll|llll}
%        & \multicolumn{3}{c}{group 1}               &    \multicolumn{4}{c}{group 2} \\ \hline
%  root  & \orth{bila} & \orth{tENa} & \orth{t'et'a}  & \orth{ambib}   & \orth{rot'}    & \orth{ts'af}   & \orth{hEd} \\
%  translation & eat        & sleep       & drink          & read           & run            & write          & go \\
%\end{tabular}
%\end{table}

\subsection{Command affixes}
The different command forms are encoded through affixes which are attached to the verb root. In the case of vowel-final verb roots, the final vowel is dropped when there is a suffix (as in (\ref{ex:command:vfinal-f})). In consonant-final verb roots, there is no suffix in the masculine imperative form, and the verb form consists of only the verb root. This can be seen in (\ref{ex:command:cfinal-m}). The hortative forms in (\ref{ex:command:vfinal-h}) and (\ref{ex:command:cfinal-h}) consist of a verbal prefix \textit{in-} or \textit{ini-} which is attached to the verb root.

\begin{multicols}{2}
\begin{exe}
  \ex\label{ex:command:vfinal-m}
    \gll \orth{tENa-$\emptyset$} \\
    sleep-2M.IMP \\
    \trans `sleep! (to a man)'

  \ex\label{ex:command:vfinal-f}
    \gll \orth{tEN-I} \\
    sleep-2F.IMP \\
    \trans `sleep! (to a woman)'

  \ex\label{ex:command:vfinal-p}
    \gll \orth{tEN-u} \\
    sleep-2P.IMP \\
    \trans `sleep! (pl)'

  \ex\label{ex:command:vfinal-h}
    \gll \orth{ini-tENa} \\
    sleep-2H.IMP \\
    \trans `let's sleep!'

  \ex\label{ex:command:cfinal-m} \gll 
    \orth{ambib-$\emptyset$} \\
    read-2M.IMP \\
    \trans `read! (to a man)'

  \ex\label{ex:command:cfinal-f} \gll 
    \orth{ambib-I} \\
    read-2F.IMP \\
    \trans `read! (to a woman)'

  \ex\label{ex:command:cfinal-p} \gll 
    \orth{ambib-u} \\
    read-2P.IMP \\
    \trans `read! (pl)'

  \ex\label{ex:command:cfinal-h} \gll 
    \orth{in-ambib} \\
    read-2H.IMP \\
    \trans `let's read'
\end{exe}
\end{multicols}

These forms are negated using the prefix \orth{at-} in the masculine, feminine, and plural forms. This can be seen for the verb \orth{mets'af} `to write' in (\ref{ex:neg-cfinal-m}), (\ref{ex:neg-cfinal-f}), and (\ref{ex:neg-cfinal-p}). In the hortative form, the prefix \orth{an-} is used in the negative form instead of \orth{in-}.

\begin{exe}
  \ex\label{ex:neg-cfinal-m} \gll 
    \orth{at-its'af-$\emptyset$} \\
    \textsc{neg}-write-\textsc{2m.imp} \\
    \trans `don't write! (to a man)'

  \ex\label{ex:neg-cfinal-f} \gll 
    \orth{at-its'af-I} \\
    \textsc{neg}-write-\textsc{2f.imp} \\
    \trans `don't write! (to a woman)'

  \ex\label{ex:neg-cfinal-p} \gll 
    \orth{at-its'af-u} \\
    \textsc{neg}-write-\textsc{2p.imp} \\
    \trans `don't write! (pl)'

  \ex\label{ex:neg-cfinal-h} \gll 
    \orth{an-its'af} \\
    \textsc{2h.imp.neg}-write \\
    \trans `let's not write'
\end{exe}

\noindent These forms are summarized in table \ref{tab:commands-affixes}.

\begin{table}[ht!]
\centering
\caption{Command affixes}
\label{tab:commands-affixes}
\begin{tabular}{l|l|l}
  form & positive & negative \\ \hline
  \textsc{2m} & \orth{-$\emptyset$} & \orth{at-, -$\emptyset$} \\
  \textsc{2f} & \orth{-I}           & \orth{at-, -I} \\
  \textsc{2p} & \orth{-u}           & \orth{at-, -u}\\
  \textsc{2h} & \textit{in-$/$\_ V, ini-$/$\_C} & \orth{an-} \\
\end{tabular}
\end{table}

\subsection{Phonological changes}
In addition to the affixes, the root vowel changes for some verbs in the imperative form. Examples of this include \textit{\orth{merot'}} `to run'  and \textit{\orth{mehEd}} `to go', which are shown in (\ref{ex:stem-merot}) and (\ref{ex:stem-mehed}).

\begin{exe}
  \ex\label{ex:stem-merot} \orth{merot'} `to run' $\rightarrow$ \orth{rut'} `run! (to a male)'
  \ex\label{ex:stem-mehed} \orth{mehEd} `to go' $\rightarrow$ \orth{hId} `go! (to a male)'
  \ex\label{ex:stem-metsaf} \orth{mets'af} `to write' $\rightarrow$ \orth{ts'af} `write! (to a male)'
\end{exe}

\noindent This process occurs in one-syllable verb roots. We can put together the following correspondences in (\ref{ex:command:v-changes}):

\begin{exe}
  \ex\label{ex:command:v-changes} \begin{enumerate}
    \item \orth{E $\rightarrow$ I}
    \item \orth{o $\rightarrow$ u}
    \item \orth{a $\rightarrow$ a} (no change)
  \end{enumerate}
\end{exe}
In the case of /\orth{E}/ and /\orth{o}/, these vowels are being raised, but we do not see a similar process with /\orth{a}/ as seen in (\ref{ex:stem-metsaf}).

In the feminine forms of many commands, the consonant before the /-\orth{I}/ suffix becomes palatalized. This can be seen in \ref{ex:palatalized1} and \ref{ex:palatalized2}, and appears to only occur with /t/ and /d/.
\begin{multicols}{2}
\begin{exe}
  \ex\label{ex:palatalized1}
  \gll \orth{t'ech'-I} \\
    drink-2F.IMP \\
    \trans `drink! (to a female)'

  \ex\label{ex:palatalized2}
  \gll \orth{hIj-I} \\
    go-2F.IMP \\
    \trans `go! (to a female)'
\end{exe}
\end{multicols}

\noindent This can be captured with the following phonological rule. This does not appear to be a general rule in the language and may be specific to commands.

\begin{exe}
  \ex voiceless stop $\rightarrow$ palatalized $/$ \_ \orth{I}
\end{exe}

\newpage
\section{Copular sentences}
\setcounter{exx}{0}

\subsection{Structure}
\iffalse
a. What is the structure and order of constituents in copular sentences? What types of complements does the 'be' element take? Is the 'be' element a verb? What forms (case, etc.) are used for the subject and complement? Be careful in this section not to confuse noun and noun phrase, adjective and adjective phrase, etc. Go ahead and make assumptions about parts of speech if it helps (e.g., noun vs. adjective).
\fi

The `be' element is the verb \orth{new} (here in the masculine form; all forms can be seen in table \ref{tab:copula}). Copular sentences have the form SOV. The complements in each of these sentences can be phrases headed either by nouns, the class of words that includes things, or by adjectives, the class of words that are used to modify nouns.  These two possibilities are shown in (\ref{ex:cop-adj}) and (\ref{ex:cop-n}). 

\begin{exe}
  \ex\label{ex:cop-adj} \gll \orth{wef-och} \orth{tinishIyE} \orth{nachew} \\
  bird-\textsc{p} small be\textbackslash\textsc{3p} \\
  \trans `The birds are small'\footnote{In these glosses, \textsc{3sf} means third person singular feminine, \textsc{p} means plural, etc.}

  \iffalse
  \ex \gll \orth{ichI} \orth{wef} \orth{tinishIyE} \orth{nat} \\
  this bird small be\textbackslash\textsc{3sf} \\
  \trans `This bird is small'
  \fi

  \ex\label{ex:cop-n} \gll \orth{Arsema} \orth{temarI} \orth{nech} \\
  Arsema student be\textbackslash\textsc{3sf} \\
  \trans{Arsema is a student}
\end{exe}

The difference between these phrases is sometimes dependent on word order.  In (\ref{ex:cop-order-adj}), \orth{k'ey} `red' is a complement of the verb. In (\ref{ex:cop-order-n}), it modifies the noun phrase, which is a complement of the \orth{new}.

\begin{multicols}{2}
\begin{exe}
  \ex\label{ex:cop-order-adj}
  \Tree [.S [.DP [.D \orth{ihE}\\this ] [.NP [.N \orth{wef}\\bird ] ] ] [.VP [.AdjP [.Adj \orth{k'ey}\\red ] ] [.V \orth{new}\\be\textbackslash\textsc{3sm} ] ] ]

  \centering
  `This bird is red'

  \ex\label{ex:cop-order-n} 
  \Tree [.S [.NP [.N \orth{ihE}\\this ] ] [.VP [.NP [.AdjP [.Adj \orth{k'ey}\\red ] ] [.N \orth{wef}\\bird ] ] [.V \orth{new}\\be\textbackslash\textsc{3sm} ] ] ]

  \centering
  `This is a red bird'
\end{exe}
\end{multicols}

The forms of the subjects and complements are the same. Amharic appears to track arguments using word order, and there are no indications of case in copular sentences.

\subsection{Copula forms}
\iffalse
b. Present tense and negative forms of the copula. Give the different person and number forms of the copula in the present tense in affirmative and negative forms:
neny ‘I am’                                                          aydelehum ‘I am not’
neh ‘you m. are’                                                  aydelehim ‘you m. are not’
If you feel you can divide these forms into morphemes, then add hyphens.
\fi

Table \ref{tab:copula} lists the forms of the copula in the postive and negative.

\begin{table}[ht]
\centering
\caption{Copula forms}
\label{tab:copula}
  \begin{tabular}{l|llll}
    form & affirmative & & negative & \\ \hline
    \textsc{1s}  & neny & `I am'          & aydelehum & `I am not'\\
    \textsc{2sm} & neh & `You (m) are'    & aydelehim & `You (m) are not'\\
    \textsc{2sf} & nesh & `You (f) are'   & aydeleshim & `You (f) are not'\\
    \textsc{3sm} & new & `He is'          & aydelem & `He is not'\\
    \textsc{3sf} & nech/nat & `She is'    & aydelechim & `She is not'\\
    \textsc{1p}  & nen & `We are'         & aydelenim & `We are not'\\
    \textsc{2p}  & nachu & `You (pl) are' & aydelachum & `You (pl) are not'\\
    \textsc{3p}  & nachew & `They are'    & aydelum & `They are not'\\
  \end{tabular}
\end{table}

The word order is SOV in both the positive and negative. The only difference between affirmative and negative is the form of the copula.

\begin{exe}
  \ex \orth{inE temarI neny} `I am a student'
  \ex \orth{ante temarI neh} `You (m) are a student'
  \ex \orth{anchI temarI nesh} `You (f) are a student'
  \ex \orth{isu temarI new} `He is a student'
  \ex \orth{iswa temarI nech} `She is a student'
  \ex \orth{inya temarIyoch nen} `We are students'
  \ex \orth{inante temarIyoch nachu} `You (pl) are students'
  \ex \orth{inesu temarIyoch nachew} `They are students'
  \ex \orth{inE temarI aydelehum} `I am not a student'
  \ex \orth{ante temarI aydelehim} `You (m) are not a student'
  \ex \orth{anchI temarI aydeleshim} `You (f) are not a student'
  \ex \orth{isu temarI aydelem} `He is not a student'
  \ex \orth{iswa temarI aydelechim} `She is not a student'
  \ex \orth{inya temarIyoch aydelenim} `We are not students'
  \ex \orth{inante temarIyoch aydelachum} `You (p) are not students'
  \ex \orth{inesu temarIyoch aydelum} `They are not students'
\end{exe}

\newpage
\section{Verbs}
\setcounter{exx}{0}

Maybe an overview here?

\subsection{Subject person marking}

In Amharic, subject person marking differs in each tense.

\subsubsection{Past positive, negative}
\iffalse
*Past tense in positive and negative.* Please describe the person marking in intransitive verbs in the past tense in the positive and negative. According to my notes, we have these forms for merot' 'to run', mezemir 'to sing', mesrat 'to work', and meblat 'to eat'. You might use a layout like this:

(1)   sera-w 'I worked'                al-sera-w-m 'I didn't work'

        sera-h 'you m. worked'      al-sera-h-im 'you m. didn't work'
\fi

In the past tense, there are different affixes depending on whether the verb root ends in a consonant or a vowel. Example (\ref{ex:past:sing}) lists the different forms for a verb ending in a consonant.

\begin{exe}
  \ex\label{ex:past:sing} \orth{mezemir} `to sing' \\
  \begin{tabular}{ll}
    \orth{zemer-ku}\footnote{Arsema was unsure if this form was correct, and also provided \orth{zemerkuwin}} `I sang' & \orth{al-zemer-ku-m} `I didn't sing' \\
    \orth{zemer-k} `You (m) sang' & \orth{al-zemer-k-im} `You (m) didn't sing \\
    \orth{zemer-sh} `You (f) sang' & \orth{al-zemer-sh-im} `You (f) didn't sing \\
    \orth{zemer-e} `He sang' & \orth{al-zemer-e-m} `He didn't sing' \\
    \orth{zemer-ech} `She sang' & \orth{al-zemer-ech-im} `She didn't sing' \\
    \orth{zemer-en} `We sang' & \orth{al-zemer-en-im} `We didn't sing' \\
    \orth{zemer-achu} `You (pl) sang' & \orth{al-zemer-achu-m} `You (pl) didn't sing' \\
    \orth{zemer-u} `They sang' & \orth{al-zemer-u-m} `They didn't sing' \\
  \end{tabular}
\end{exe}

\noindent By contrast, (\ref{ex:past:work}) shows the forms for a verb root ending in a consonant.

\begin{exe}
  \ex\label{ex:past:work} \orth{mesrat} `to work' \\
  \begin{tabular}{ll}
    \orth{sera-w} `I worked' & \orth{al-sera-w-m} `I didn't work' \\
    \orth{sera-h} `You (m) worked' & \orth{al-sera-h-im} `You (m) didn't work \\
    \orth{sera-sh} `You (f) worked' & \orth{al-sera-sh-im} `You (f) didn't work \\
    \orth{sera} `He worked' & \orth{al-sera-m} `He didn't work' \\
    \orth{sera-ch} `She worked' & \orth{al-sera-ch-im} `She didn't work' \\
    \orth{sera-n} `We worked' & \orth{al-sera-n-im} `We didn't work' \\
    \orth{sera-chu} `You (pl) worked' & \orth{al-sera-chu-m} `You (pl) didn't work' \\
    \orth{ser-u} `They worked' & \orth{al-ser-u-m} `They didn't work' \\
  \end{tabular}
\end{exe}

For both types of verbs, the past negative is formed with the \orth{al-} prefix and the \orth{-im} suffix. All V-final forms preserve the final root vowel with the exception of the third-person plural form \orth{seru}. In this form, the final vowel is dropped and replaced with the \orth{-u} suffix; otherwise, it would be indistinguishable from the \textsc{3sm} form. These two paradigms are summarized in table \ref{tab:past:subject-affixes}.

\begin{table}[ht!]
\centering
\caption{Past subject affixes}
\label{tab:past:subject-affixes}
  \begin{tabular}{l|ll}
    form & C-final & V-final \\ \hline
    \textsc{1s}  & \orth{-ku}   & \orth{-w} \\ 
    \textsc{2sm} & \orth{-k}    & \orth{-h} \\
    \textsc{2sf} & \orth{-sh}   & \orth{-sh} \\
    \textsc{3sm} & \orth{-e}    & \orth{-$\emptyset$} \\
    \textsc{3sf} & \orth{-ech}  & \orth{-ch} \\
    \textsc{1p}  & \orth{-en}   & \orth{-n} \\
    \textsc{2p}  & \orth{-achu} & \orth{-chu} \\
    \textsc{3p}  & \orth{-u}    & \orth{-u}
  \end{tabular}
\end{table}

%Table \ref{tab:negatives-run} shows the past affirmative and negative forms of \orth{merot'} `to run'.
%
%\begin{table}[ht]
%\centering
%\caption{Postive and negative forms of `to run'}
%\label{tab:negatives-run}
%\begin{tabular}{llll}
%  \orth{inE rot'-kuwin} & `I ran'        & \orth{inE al-rot'-ku-m} & `I didn't run' \\
%  \orth{ante rot'-k} & `You (m) ran'     & \orth{ante al-rot'-k-im} & `You (m) didn't run' \\
%  \orth{anchI rot'-sh} & `You (f) ran'   & \orth{anchIma al-rot'-sh-im} & `You (f) didn't run' \\
%  \orth{isu rot'-e} & `He ran'           & \orth{isu al-rot'-e-m} & `He didn't run' \\
%  \orth{iswa rot'-ech} & `She ran'       & \orth{iswa al-rot'-ech-im} & `She didn't run' \\
%  \orth{inya rot'-en} & `We ran'         & \orth{inya al-rot'-in-im} & `We didn't run' \\
%  \orth{inante rot'-achu} & `You (pl) ran' & \orth{inante al-rot'-achu-m} & `You (pl) didn't run' \\
%  \orth{inesu rot'-u} & `They ran'         & \orth{inesu al-rot'-u-m} & `They didn't run' \\
%\end{tabular}
%\end{table}
%
%Here, we see that the negative forms are formed by the prefix \orth{al-} and an \orth{-im} suffix which occurs after the person marking suffix.

\subsubsection{Present/future positive, negative}
\iffalse
*Present/future tense in positive and negative.* Please describe the person marking in intransitive verbs in the present/future tense in the positive and negative. I think we only have these forms for merot' 'to run', collected on 2/16 (irot'alew 'I run', alrot'im 'I don't run'). These forms are surprising and quite challenging.
\fi

In the present tense, there are several different forms for the positive and negative. Each form requires a prefix and a suffix, which is summarized in table \ref{tab:pres:affixes}. The different forms of \orth{merot'} are shown in (\ref{ex:pres:run}).

\begin{exe}
  \ex\label{ex:pres:run} \orth{merot'} `to run' \\
  \begin{tabular}{ll}
    \orth{inE i-rot'-al-ew} `I run'              & \orth{inE al-rot'-im} `I don't run' \\
    \orth{ante ti-rot'-al-eh} `You (m) run'      & \orth{ante at-rot'-im} `You (m) don't run \\
    \orth{anchI ti-roch'-al-esh} `You (f) run'   & \orth{anchI at-roch'-im} `You (f) don't run \\
    \orth{isu yI-rot'-al-e} `He run'             & \orth{isu ay-rot'-im} `He doesn't run' \\
    \orth{iswa ti-rot'-al-ech} `She run'         & \orth{iswa at-rot'-im} `She doesn't run' \\
    \orth{inya in-rot'-al-en} `We run'           & \orth{inya an-rot'-im} `We don't run' \\
    \orth{inante ti-rot'-al-achu} `You (pl) run' & \orth{inante at-rot'-um} `You (pl) don't run' \\
    \orth{inesu yI-rot'-al-u} `They run'         & \orth{inesu ay-rot'-um} `They don't run' \\
  \end{tabular}
\end{exe}

In the positive forms, \orth{-al} occurs after the verb root and may represent the present tense. The four different prefixes separate the forms into four different groups, as shown in table \ref{tab:pres:affixes}. This grouping seems arbitrary; it puts all 2nd person forms in one group along with the 3rd person feminine (group 2), while other third person forms are grouped (group 3) and first person forms are not grouped (groups 1 and 4).

\begin{table}[ht]
\centering
\caption{Present/future affixes}
\label{tab:pres:affixes}
  \begin{tabular}{l|lll}
    form & positive & negative & group \\ \hline
    \textsc{1s}  & \orth{i- -al-ew}    & \orth{al- -im} & 1 (\orth{i-}/\orth{al-})\\ 
    \textsc{2sm} & \orth{ti- -al-eh}   & \orth{at- -im} & 2 (\orth{ti-}/\orth{at-})\\
    \textsc{2sf} & \orth{ti- -al-esh}  & \orth{at- -im} + palatalization & 2 (\orth{ti-}/\orth{at-})\\
    \textsc{3sm} & \orth{yI- -al-e}    & \orth{ay- -im} & 3 (\orth{yI-}/\orth{ay-})\\
    \textsc{3sf} & \orth{ti- -al-ech}  & \orth{at- -im} & 2 (\orth{ti-}/\orth{at-})\\
    \textsc{1p}  & \orth{in- -al-en}   & \orth{an- -im} & 4 (\orth{in-}/\orth{an-})\\
    \textsc{2p}  & \orth{ti- -al-achu} & \orth{at- -um} & 2 (\orth{ti-}/\orth{at-})\\
    \textsc{3p}  & \orth{yI- -al-u}    & \orth{ay- -um} & 3 (\orth{yI-}/\orth{ay-}) 
  \end{tabular}
\end{table}

These sentences may then be glossed like in (\ref{ex:merot:pres:gloss1}) and (\ref{ex:merot:pres:gloss2}), where \textsc{g1} represents the group 1 prefix. 

\begin{multicols}{2}
\begin{exe}
  \ex\label{ex:merot:pres:gloss1} \gll \orth{inE} \orth{i-rot'-al-ew} \\
  I \textsc{g1.pres}-run-\textsc{pres}-\textsc{1s} \\
  \trans `I run'

  \ex\label{ex:merot:pres:gloss2} \gll \orth{ante} \orth{ti-rot'-al-eh} \\
  \textsc{2sm} \textsc{g2.pres}-run-\textsc{pres}-\textsc{2sm} \\
  \trans `You run'
\end{exe}
\end{multicols}

The negative forms of each have a prefix that changes depending on which group the form is in, and the person marker is dropped. Each of these prefixes may be a variation of the \orth{al-} prefix, which is found in the past tense, and just like the past tense, each form has the \orth{-im} suffix for negation. The prefixes represent the group of the person marking, the present tense, and negation, as shown in (\ref{ex:merot:pres:neg-gloss1}) and (\ref{ex:merot:pres:neg-gloss2}).

\begin{multicols}{2}
\begin{exe}
  \ex\label{ex:merot:pres:neg-gloss1} \gll \orth{inE} \orth{al-rot'-im} \\
  I \textsc{g1.pres.neg}-run-\textsc{neg} \\
  \trans `I don't run'

  \ex\label{ex:merot:pres:neg-gloss2} \gll \orth{ante} \orth{at-rot'-im} \\
  \textsc{2sm} \textsc{g2.pres.neg}-run-\textsc{neg} \\
  \trans `You don't run'
\end{exe}
\end{multicols}

\subsection{Object person marking in transitive verbs}
\iffalse
*Object person marking.* Please describe the object person marking in transitive verbs in the past tense (i.e., for verbs that mark the person of the subject and object). You should give a table of the suffixes and then verb forms that provide evidence for the table. You don't need to give 8 x 8 verb forms: just enough to demonstrate the object person markers and at least some subject markers to show how they interact.

Please be sure to mention what "object" means: does the verb agree with the direct object, the indirect object, both, or something else?
\fi

In transitive verbs, the direct object is marked by a verbal suffix which occurs after the subject suffix (listed in table \ref{tab:pres:affixes}). In this way, the verb agrees with both the subject and the direct object. (\ref{ex:makif:obj}) shows the paradigm for the verb \orth{mak'if} `to hug.'

\begin{exe}
  \ex\label{ex:makif:obj} \orth{mak'if} `to hug' \\
  \begin{tabular}{ll}
    \orth{ak'if-e-ny}  & `He hugged me' \\
    \orth{ak'if-e-h}   & `He hugged you (m)' \\
    \orth{ak'if-e-sh}  & `He hugged you (f)' \\
    \orth{ak'if-e-w}   & `He hugged him' \\
    \orth{ak'if-at}    & `He hugged her' \\
    \orth{ak'if-e-n}   & `He hugged us' \\
    \orth{ak'if-achu}  & `He hugged you (pl)' \\
    \orth{ak'if-achew} & `He hugged them' 
  \end{tabular}
\end{exe}

There are several phonological processes that change the surface forms of the affixes. For example, in (\ref{ex:makif:phon1}), the third-person masculine singular subject is dropped when the object affix begins with a vowel. Additionally, in (\ref{ex:makif:phon2}), the affix occurs with an initial \orth{w} which does not occur in other forms. This \orth{w} may be inserted after consonants in the \textsc{3sf}, \textsc{2p}, and \textsc{3p} forms, but this is blocked by the underlying \orth{-e} in (\ref{ex:makif:phon1}).

\begin{multicols}{2}
\begin{exe}
  \ex\label{ex:makif:phon1} \gll \orth{ak'if-$\emptyset$-at} \\
  hug-\textsc{3sm.subj}-\textsc{3sf.obj} \\
  \trans `He hugged her'

  \ex\label{ex:makif:phon2} \gll \orth{ak'if-ach-wat} \\
  hug-\textsc{2p.subj}-\textsc{3sf.obj} \\
  \trans `You (pl) hugged her'
\end{exe}
\end{multicols}

These forms are summarized in table \ref{tab:person-marking}.

\begin{table}[ht]
\centering
\caption{Subject and object suffixes}
\label{tab:person-marking}
  \begin{tabular}{l|ll}
    form &  object \\ \hline
    \textsc{1s}  & \orth{-ny} \\ 
    \textsc{2sm} & \orth{-h} \\
    \textsc{2sf} & \orth{-sh} \\
    \textsc{3sm} & \orth{-t, -w} \\
    \textsc{3sf} & \orth{-(w)at} \\
    \textsc{1p}  & \orth{-en} \\
    \textsc{2p}  & \orth{-(w)achu} \\
    \textsc{3p}  & \orth{-(w)achew}
  \end{tabular}
\end{table}

Additionally, we have evidence that suggests that indirect objects work similarly. In these examples, the direct object comes before the verb, and the indirect object occurs as a verbal suffix. The verb agrees with the subject and the indirect object in (\ref{ex:mestet:indirect}).

\begin{exe}
  \ex\label{ex:mestet:indirect} \orth{mest'et} `to give' \\
  \begin{tabular}{ll}
    \orth{mets'af set'-ech-iny} & `She gave me a book' \\
    \orth{mets'af set'-ech-ih} & `She gave you (m) a book' \\
    \orth{mets'af set'-ech-ish} & `She gave you (f) a book' \\
    \orth{mets'af set'-ech-u} & `She gave him a book' \\
    \orth{mets'af set'-ech-at} & `She gave her a book' \\
    \orth{mets'af set'-ech-in} & `She gave us a book' \\
    \orth{mets'af set'-ech-achu} & `She gave you (pl) a book' \\
    \orth{mets'af set'-ech-achew} & `She gave them a book'
  \end{tabular}
\end{exe}

\noindent In several of these forms, \orth{i} is inserted before the object prefix. This is common throughout the language, and we would expect \orth{i} to be inserted as the default vowel.

\newpage
\section{Causatives}
\setcounter{exx}{0}
\iffalse
1. We have seen a few examples of causatives in Amharic:

kiflén ats’edawt ‘I clean my room’
inaté kiflén asts’edachíny ‘My mom made me clean my room’
 
iné arigkut ‘I did it’
iswa asderigechíny ‘she made me do it’ [I don't know what this "de-" is]
 
iné k’omkuwin ‘I stopped’
polísu ask’omeny ‘the policeman caused me to stop’
Describe what is going on in these pairs of examples: what is happening to the verb? how is agreement working (what is the verb agreeing with)? what is getting case marked as an object? To make your assumptions clear, draw a tree for one non-causative sentence and one causative counterpart.
\fi

Causatives are marked with the verbal prefix \orth{as-}. In the causative form, the verb agrees with the causer (in the subject position) and the causee (in the object position). This is different from the non-causative form, in which the verb agrees with the subject and the object of the sentence, as seen in (\ref{ex:caus:clean1}) with the verb \orth{mats'dat} `to clean.' However, the object is marked with the object suffix \orth{-n} in both forms. We can see the causative form in (\ref{ex:caus:clean2}).

\iffalse
\begin{exe}
  \ex \orth{mats'dat} `to clean' \\
      \orth{kiflEn ats'edawt} `I clean my room' \\
      \orth{inatE kiflEn asts'edachIny} `My mom made me clean my room'

  \ex \orth{madrig} `to do' \\
      \orth{inE arigkut} `I did it' \\
      \orth{iswa asderigechIny} `She made me do it'
\end{exe}
\fi

\begin{exe}
  \ex\label{ex:caus:clean1} \gll \orth{kifl-E-n} \orth{ats'eda-w-t} \\
  room-\textsc{1s.poss-obj} clean-\textsc{1s.subj-3sm.obj} \\
  \trans `I cleaned my room'

  \ex\label{ex:caus:clean2} \gll \orth{inat-E} \orth{kifl-E-n} \orth{as-ts'eda-ch-iny} \\
  mom-\textsc{1s.poss} room-\textsc{1s.poss-obj} \textsc{caus}-clean-\textsc{3sf.subj-1s.obj} \\
  \trans `My mom made me clean my room'
\end{exe}

\noindent (\ref{ex:caus:tree1}) and (\ref{ex:caus:tree2}) show the structures of both sentences.

\begin{exe}
  \ex\label{ex:caus:tree1} \Tree [.S [.NP $\emptyset$ ] [.VP [.NP N\\\orth{kifl-E-n}\\room-\textsc{1s.poss-obj} ] V\\\orth{ats'eda-w-t}\\clean-\textsc{1s.subj-3sm.obj} ] ]

  \ex\label{ex:caus:tree2} \Tree [.S [.NP N\\\orth{inat-E}\\mom-\textsc{1s.poss} ] [.VP [.NP N\\\orth{kifl-E-n}\\room-\textsc{1s.poss-obj} ] V\\\orth{as-ts'eda-ch-iny}\\\textsc{caus}-clean-\textsc{3sf.subj-1s.obj} ] ]
\end{exe}


\newpage
\section{Dependent clauses}
\setcounter{exx}{0}

TODO: Add an intro here.

\subsection{Complement clauses}

There are several types of complement clauses in Amharic.

\subsubsection{Null complementizer}

In some cases, complement clauses are formed without an explicit complementizer. This can be seen in (\ref{ex:comp:null1}) and (\ref{ex:comp:null2}).
\begin{exe}
  \ex\label{ex:comp:null1} \gll \orth{i-met'-ale-w} \orth{belo-$\emptyset$-ny-al} \\
  \textsc{g1.pres}-come-\textsc{pres-1s} said-\textsc{3sm.subj}-\textsc{1s.obj}-al \\
  \trans `He said, ``I am coming'''\footnote{The meaning of this -al suffix is unknown.}

  \ex\label{ex:comp:null2} \gll \orth{k'onjo} \orth{nat} \orth{t'eye-kuwin} \\
  pretty be\textbackslash\textsc{3sf} asked-\textsc{1s} \\
  ```Is she pretty?'' I asked'
\end{exe}

\noindent In these sentences, the CP is the object of the main verb. These sentences have the following structure:

\begin{exe}
\ex \Tree [.S [.NP N\\$\emptyset$ ] [.VP [.CP [.S [.NP N\\$\emptyset$ ] [.VP V\\\orth{imet'alew} ] ] C\\$\emptyset$ ] V\\\orth{belonyal} ] ]
\end{exe}


\subsubsection{Complementizer prefixes}

There are several examples of verbal prefixes that function as complementizers. In many of these examples,  \orth{inde-} is used; this is shown in (\ref{ex:comp:inde1}) and (\ref{ex:comp:inde2}). 
\begin{exe}
  \ex\label{ex:comp:inde1} \gll \orth{inde-mI-met'a} \orth{belo-$\emptyset$-ny-al} \\
  \textsc{comp}-\textsc{3sm.comp}-come\textbackslash\textsc{3sm.past} said-\textsc{3sm.subj}-\textsc{1s.obj}-al \\
  \trans `He said that he's coming'

  \ex\label{ex:comp:inde2} \gll \orth{temarI} \orth{inde-hon-ku} \orth{tenager-ku} \\
  student \textsc{comp}-be-\textsc{1s.subj} said-\textsc{1s.subj} \\
  \trans `I said that I am a student'
\end{exe}

\noindent Another such prefix is \orth{ke-}, which functions similarly in (\ref{ex:comp:ke}). 

\begin{exe}
  \ex\label{ex:comp:ke} \gll \orth{k'onjo} \orth{ke-hon-ech} \orth{t'eye-kuwin} \\
  pretty \textsc{comp}-be-\textsc{3sf} asked-\textsc{1s} \\
  \trans `I asked if she was pretty'
\end{exe}

In some examples, these prefixes are accompanied by a variation of \orth{-mi-} and even a verbal suffix in the \textsc{1sf}, \textsc{2p}, and \textsc{3p} forms. It is unclear when these are used, as they occur in forms like (\ref{ex:comp:inde1}) but not in forms like (\ref{ex:comp:inde2}). This paradigm is shown in Table \ref{tab:comp:forms}.

\begin{table}[ht!]
\centering
\caption{Complementizer forms}
\label{tab:comp:forms}
\begin{tabular}{lll}
  \textsc{1s} & \orth{meblat inde-mi-felig} & `that I want to eat' \\
  \textsc{2sm} & \orth{meblat inde-mit-felig} & `that you (m) want to eat' \\
  \textsc{2sf} & \orth{meblat inde-mit-felig-I} & `that you (f) want to eat' \\
  \textsc{3sm} & \orth{meblat inde-mI-felig} & `that he wants to eat' \\
  \textsc{3sf} & \orth{meblat inde-mit-felig} & `that she wants to eat' \\
  \textsc{1p} & \orth{meblat inde-mini-felig} & `that we want to eat' \\
  \textsc{2p} & \orth{meblat inde-miti-felig-u} & `that you (pl) want to eat' \\
  \textsc{3p} & \orth{meblat inde-felig-u} & `that they want to eat' 
\end{tabular}
\end{table}

(\ref{ex:comp:inde:tree}) shows the structure of these sentences; the second tree shows how the verb moves up to the complementizer \orth{inde-}. As in the previous examples, the CP is the object of the verb \orth{belonyal}.

\begin{exe}
  \ex\label{ex:comp:inde:tree} \Tree [.S [.NP N\\$\emptyset$ ] [.VP [.CP [.S [.NP N\\$\emptyset$ ] [.VP V\\\orth{mImet'a} ] ] C\\\orth{inde-} ] V\\\orth{belonyal} ] ]
\Tree [.S [.NP N\\$\emptyset$ ] [.VP [.CP [.S [.NP N\\$\emptyset$ ] [.VP V\\$t$ ] ] C\\\orth{inde-mImet'a} ] V\\\orth{belonyal} ] ]
\end{exe}

\subsubsection{Infinitives}

Infinitives are formed with the verbal prefix \orth{me-} and function like other prefix complementizers. This is shown in (\ref{ex:inf1}), (\ref{ex:inf2}), and (\ref{ex:inf3}).

\begin{exe}
  \ex\label{ex:inf1} \gll \orth{me-blat} \orth{i-felig-ale-w} \\
  \textsc{inf}-eat \textsc{g1.pres}-want-\textsc{pres-1s} \\
  \trans `I want to eat'

  \ex\label{ex:inf2} \gll \orth{me-blat} \orth{inde-mIfelig} \orth{negro-$\emptyset$-nya} \\
  \textsc{inf}-eat \textsc{comp}-want told-\textsc{3sm.subj-1s.obj} \\
  \trans `He told me that he wants to eat'

  \ex\label{ex:inf3} \gll \orth{ruz} \orth{me-blat} \orth{ti-felig-achu} \orth{wey?} \\
  rice \textsc{inf}-eat \textsc{g2.pres}-want-\textsc{2p} right? \\
  `You all want to eat rice, right?'
\end{exe}

The structure of these sentences is shown in (\ref{ex:inf:tree}). The verb \orth{blat} `eat' moves up from the V to the C (\orth{me-}).

\begin{exe}
  \ex\label{ex:inf:tree} \Tree [.S [.NP N\\$\emptyset$ ] [.VP [.CP [.S [.NP N\\$\emptyset$ ] [.VP [.NP N\\\orth{ruz} ] V\\$t$ ] ] C\\\orth{me-blat} ] V\\\orth{tifeligachu} ] ]
\end{exe}

\subsection{Relative clauses}

Relative clauses in Amharic use the relativizer \orth{ye-} and function like other CPs in the language. Since Amharic is a head-final language, relative clauses occur before the nouns they modify. In subject relative clauses, the verb in the relative clause agrees with the noun that the clause modifies as the subject and the object within the clause as the object. This can be seen in (\ref{ex:rel:gll_subj}).

\begin{exe}
  \ex\label{ex:rel:gll_subj} \gll \orth{ye-lebes-ke-w} \orth{libs} \orth{arIf} \orth{new} \\
  \textsc{rel}-wear-\textsc{2sm.s-3s.o} clothes nice be\textbackslash\textsc{3s} \\
  \trans `What you are wearing is nice'
\end{exe}

The structure of this sentence can be seen in (\ref{ex:rel:tree_subj}). There is a gap at the subject of the embedded CP, and the verb moves up to the complementizer.

\begin{exe}
  \ex\label{ex:rel:tree_subj} \Tree [.S 
    [.NP
      [.CP 
        [.S 
          [.NP $t$ ]
          [.VP $t$ ]
        ]
        C\\\orth{ye-lebeskew}
      ] 
      N\\\orth{libs}
    ]
    [.VP 
      [.AdjP Adj\\\orth{arIf} ]
      V\\\orth{new}
    ]
  ] 
\end{exe}

In object relative clauses, the relativized noun is the direct object. The verb in the relative clause agrees with the noun it modifies as the object and the subject of the relative clause as the subject. This can be seen in (\ref{ex:rel:gll_obj}).

\begin{exe}
  \ex\label{ex:rel:gll_obj} \gll \orth{ye-sera-w-t} \orth{buna} \orth{yet} \orth{new} \\
  \textsc{rel}-make-\textsc{1s.s-3s.o} coffee where be\textbackslash\textsc{3s} \\
  \trans `Where is the coffee I made'
\end{exe}

The structure of this sentence can be seen in (\ref{ex:rel:tree_obj}). There is a gap at the object of the relative clause, and the verb moves up to the complementizer.

\begin{exe}
  \ex\label{ex:rel:tree_obj} \Tree [.S 
    [.NP
      [.CP 
        [.S 
          [.NP N\\$\emptyset$ ]
          [.VP
            [.NP N\\$t$ ]
            V\\$t$
          ]
        ]
        C\\\orth{ye-serawt}
      ] 
      N\\\orth{buna}
    ]
    [.VP 
      [.AdvP Adv\\\orth{yet} ]
      V\\\orth{new}
    ]
  ] 
\end{exe}

\end{document}

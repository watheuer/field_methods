\documentclass[12pt]{article}

% packages
\usepackage{tipa}
\usepackage{enumerate}
\usepackage{linguex}
\usepackage{gb4e}
\noautomath
\usepackage{xstring}
\usepackage{multirow}

% define \phon for phonetic examples
\newcommand{\phon}[1]{$[$\textipa{#1}$]$}
\newcommand{\orth}[1]{\StrSubstitute{#1}{I}{\'{i}}[\x]\StrSubstitute{\x}{E}{\'{e}}[\x]\StrSubstitute{\x}{N}{\~{n}}[\x]\x}

% margins and spacing
\usepackage[letterpaper, margin=1in]{geometry}

\iffalse
HW 3 Commands

First, please review comments on your last assignment so you don't keep doing the same thing. There is an example paper under Course Documents called Lee-Final that is a good example of style.

Write up a description of commands (positive to male, female, plural group; negative to male, female, plural group; hortatives ("let's" expressions) in positive and negative). You might want to do this in terms of the infinitive.

It's possible that some of the affixes are person markers: do your best for now and you can edit this section as you learn more.

You'll want to say something about palatalization in commands to women. A central question is whether this process is a general phonological rule in the language or specific to commands. Do your best to describe all the consonants that palatalization applies to.

From now on you'll want to start using three-line format as much as possible:

(1)  K'um-u!
     stand-IMP.PL
     'Stand! (to pl.)'

You might make some wrong guesses about the meanings of affixes, but it will get you started.
\fi


\begin{document}

\begin{center}
{\Large Homework 3} \\
{\large Will Theuer\footnote{Collaborated with Julia Ruth}}
\end{center}


\section{Orthography}
The following orthography was created for easier transcription of Amharic and is used in the rest of the paper.



\section{Commands}
There are eight different command forms in Amharic. In the second person, there positive and negative forms for male, female, and plural recipients. Additionally, there are both positive and negative hortative forms.

\begin{table}[ht]
\centering
\caption{Command forms}
\label{tab:commands-data}
\begin{tabular}{l|lll|llll}
     & \multicolumn{3}{c}{group 1}                                          & \multicolumn{4}{c}{group 2} \\
     & \orth{meblat} & \orth{metENat} & \orth{met'et'at}  & \orth{mambib}   & \orth{merot'}    & \orth{mets'af}   & \orth{mehEd} \\ \hline
  F  & \orth{bI}     & \orth{tENI}    & \orth{t'ech'I}    & \orth{ambibI}   & \orth{ruch'I}    & \orth{ts'afI}    & \orth{hIjI} \\
  M  & \orth{bila}   & \orth{tENa}    & \orth{t'et'a}     & \orth{ambib}    & \orth{rut'}      & \orth{ts'af}     & \orth{hId} \\
  P  & \orth{bilu}   & \orth{tENu}    & \orth{t'et'u}     & \orth{ambibu}   & \orth{rut'u}     & \orth{ts'afu}    & \orth{hIdu} \\
  H  & \orth{inibla} & \orth{initENa} & \orth{init'eta}   & \orth{inambib}  & \orth{inirut'}   & \orth{inits'af}  & \orth{inihId} \\
  NF & \orth{atibI}  & \orth{atitENI} & \orth{atit'ech'I} & \orth{atambibI} & \orth{atiruch'I} & \orth{atits'afI} & \orth{atihIjI} \\
  NM & \orth{atibla} & \orth{atitENa} & \orth{atit'et'a}  & \orth{atambib}  & \orth{atirut'}   & \orth{atits'af}  & \orth{atihId} \\
  NP & \orth{atiblu} & \orth{atitENu} & \orth{atit'et'u}  & \orth{atambibu} & \orth{atirut'u}  & \orth{atits'afu} & \orth{atihIdu} \\
  NH & \orth{anibla} & \orth{anitENa} & \orth{anit'et'a}  & \orth{anambib}  & \orth{anirut'}   & \orth{anits'af}  & \orth{anihId} \\
\end{tabular}
\end{table}

\subsection{Roots}

To analyze these command forms, it is important to recognize the verb root. Each infinitive form starts with the prefix /me-/. In some of these forms, the /e/ is dropped when the root begins with a vowel.

\begin{exe}
  \ex \gll me-ambib $>$ \phon{mambib} \\
      INF-read \\
      \trans `to read'
\end{exe}

\noindent Verbs are broken up into two different groups based on their root structure. Group 1 nouns (as seen in table \ref{tab:commands-data}) have roots consisting of two syllables where the second syllable is CV. These nouns also have a word-final /-t/ indicating the infinitive. 

\begin{exe}
  \ex \gll \orth{me-tENa-t} \\
      INF-sleep-INF \\
      \trans `to sleep'
\end{exe}

\noindent The language may require verbs to end in a consonant in their infinitive forms. Therefore, verbs in group 1 take an extra /-t/, while the verbs in group 2 already have a final consonant as part of the verb root. The verbs from table \ref{tab:commands-data} must have the following roots:

\begin{table}[ht]
\centering
\caption{Verb roots}
\label{tab:commands-roots}
\begin{tabular}{l|lll|llll}
        & \multicolumn{3}{c}{group 1}               &    \multicolumn{4}{c}{group 2} \\ \hline
  root  & \orth{bla} & \orth{tENa} & \orth{t'et'a}  & \orth{ambib}   & \orth{rot'}    & \orth{ts'af}   & \orth{hEd} \\
  trans & eat        & sleep       & drink          & read           & run            & write          & go \\
\end{tabular}
\end{table}

\subsection{Command affixes}
The different command forms are encoded through affixes which are attached to the verb root.

\begin{table}[h]
\centering
\caption{Command affixes}
\label{tab:commands-affixes}
\begin{tabular}{l|l}
  form & affix\\ \hline
  F  & \orth{-I} \\
  M  & \orth{-$\emptyset$}\\
  P  & \orth{-u}\\
  H  & \orth{in- -a}\\
  NF & \orth{at- -i}\\
  NM & \orth{at- -$\emptyset$}\\ 
  NP & \orth{at- -u}\\
  NH & \orth{an- -a}\\
\end{tabular}
\end{table}

In the case of group 1 verbs, the final vowel is dropped when there is a suffix. However, the final vowel of group 2 verbs is preserved in the masculine forms, where there is no additional suffix.

\begin{exe}
  \ex \gll \orth{tENa-$\emptyset$} \\
      sleep-2M.IMP \\
      \trans `sleep! (to a man)'

  \ex \gll \orth{ambib-$\emptyset$} \\
      read-2M.IMP \\
      \trans `read! (to a man)'
\end{exe}

\end{document}

% full table of command data
\begin{tabular}{l|llllllllll}
  verb       & \orth{meblat} & \orth{metENat} & \orth{met'et'at}  & \orth{mambib}   & \orth{mawrat} & \orth{merot'}    & \orth{mek'om}   & \orth{mek'emit'}    & \orth{mets'af}   & \orth{mehEd} \\ \hline
  you (f)    & \orth{bI}     & \orth{tENI}    & \orth{t'ech'I}    & \orth{ambibI}   & \orth{awrI}   & \orth{ruch'I}    & \orth{k'umI}    & \orth{k'uch'bE}     & \orth{ts'afI}    & \orth{hIjI} \\
  you (m)    & \orth{bila}   & \orth{tENa}    & \orth{t'et'a}     & \orth{ambib}    & \orth{awr}    & \orth{rut'}      & \orth{k'um}     & \orth{k'uch'bel}    & \orth{ts'af}     & \orth{hId} \\
  you (pl)   & \orth{bilu}   & \orth{tENu}    & \orth{t'et'u}     & \orth{ambibu}   & \orth{awru}   & \orth{rut'u}     & \orth{k'umu}    & \orth{k'uch'belu}   & \orth{ts'afu}    & \orth{hIdu} \\
  let's      & \orth{inibla} & \orth{initENa} & \orth{init'eta}   & \orth{inambib}  & \orth{inawra} & \orth{inirut'}   & \orth{inik'um}  & \orth{inik'emit'}   & \orth{inits'af}  & \orth{inihId} \\
  don't (f)  & \orth{atibI}  & \orth{atitENI} & \orth{atit'ech'I} & \orth{atambibI} & \orth{atawrI} & \orth{atiruch'I} & \orth{atik'umI} & \orth{atik'emich'I} & \orth{atits'afI} & \orth{atihIjI} \\
  don't (m)  & \orth{atibla} & \orth{atitENa} & \orth{atit'et'a}  & \orth{atambib}  & \orth{atawr}  & \orth{atirut'}   & \orth{atik'um}  & \orth{atik'emit'}   & \orth{atits'af}  & \orth{atihId} \\
  don't (pl) & \orth{atiblu} & \orth{atitENu} & \orth{atit'et'u}  & \orth{atambibu} & \orth{atawru} & \orth{atirut'u}  & \orth{atik'umu} & \orth{atik'emit'u}  & \orth{atits'afu} & \orth{atihIdu} \\
  let's not  & \orth{anibla} & \orth{anitENa} & \orth{anit'et'a}  & \orth{anambib}  & \orth{anawra} & \orth{anirut'}   & \orth{anik'um}  & \orth{anik'emit'}   & \orth{anits'af}  & \orth{anihId} \\
\end{tabular}

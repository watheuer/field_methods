\documentclass[12pt]{article}

% packages
\usepackage{tipa}
\usepackage{enumerate}
\usepackage{linguex}
\usepackage{gb4e}
\noautomath
\usepackage{xstring}
\usepackage{multirow}
\usepackage{multicol}
\usepackage{footnote}
\usepackage{qtree}
\usepackage{tree-dvips}
\makesavenoteenv{tabular}
\makesavenoteenv{table}

% define \phon for phonetic examples
\newcommand{\phon}[1]{$[$\textipa{#1}$]$}
\newcommand{\orth}[1]{\textit{\StrSubstitute{#1}{I}{\'{i}}[\x]\StrSubstitute{\x}{E}{\'{e}}[\x]\StrSubstitute{\x}{N}{\~{n}}[\x]\x}}

% margins and spacing
\usepackage[letterpaper, margin=1in]{geometry}
% =================================================================================================

\begin{document}

\begin{center}
{\Large Homework 9} \\
{\large Will Theuer}
\end{center}

\iffalse
Please describe relative clauses in Amharic.

This section would benefit from trees. Relative clauses are generally assumed to be CPs that modify a head noun. In a head-final language with externally-headed relative clauses, you'd normally expect the structure [np CP N]. You'd normally expect CP to expand as [ S C]. Remember that in a sentence like '[The student who I saw __] is my friend', the subject is the whole bracketed expression (not 'the student'). What's tricky about relative clauses is you have to describe the role of the "relativized noun" in the relative clause (as subject, direct object, etc.) and the role of the NP the relative clause is part of within the matrix sentence. In  '[The student who I saw __] is my friend', the relativized noun phrase is a direct object (indicated by the gap). The NP the relative clause modifies functions as a subject in the higher clause. Here is a link to some phrase markers for relative clauses in English: http://linguistics.stackexchange.com/questions/7089/relative-clauses-in-x-bar.

Grading rubric. 1. Did you describe the morphological patterns in relative clauses, placement of the head, form of the relative clause, type of relative clause, etc.? 2. Were you explicit about the structures (trees) you're assuming? 3. Did you gloss examples, divide them into morphemes, use tabs between words, etc.? 4. Did you give enough examples so that reader would know how to construct examples?
\fi

\section{Relative clauses}

Relative clauses in Amharic use the relativizer \orth{ye-} and work like other CPs in the language. In subject relative clauses, the verb agrees with the subject that the clause modifies and the object within the clause. This can be seen in (\ref{ex:rel:gll_subj}).

\begin{exe}
  \ex\label{ex:rel:gll_subj} \gll \orth{ye-lebes-ke-w} \orth{libs} \orth{arIf} \orth{new} \\
  \textsc{rel}-wear-\textsc{2sm.s-3s.o} clothes nice be\textbackslash\textsc{3s} \\
  \trans `What you are wearing is nice'
\end{exe}

The structure of this sentence can be seen in (\ref{ex:rel:tree_subj}). There is a gap at the subject of the embedded CP, and the verb moves up to the complementizer.

\begin{exe}
  \ex\label{ex:rel:tree_subj} \Tree [.S 
    [.NP
      [.CP 
        [.S 
          [.NP $t$ ]
          [.VP $t$ ]
        ]
        C\\\orth{ye-lebeskew}
      ] 
      N\\\orth{libs}
    ]
    [.VP 
      [.AdjP Adj\\\orth{arIf} ]
      V\\\orth{new}
    ]
  ] 
\end{exe}

In object relative clauses, there is a gap at the object of the relative clause. This can be seen in (\ref{ex:rel:gll_obj}).

\begin{exe}
  \ex\label{ex:rel:gll_obj} \gll \orth{ye-sera-w-t} \orth{buna} \orth{yet} \orth{new} \\
  \textsc{rel}-make-\textsc{1s.s-3s.o} coffee where be\textbackslash\textsc{3s} \\
  \trans `Where is the coffee I made'
\end{exe}

The structure of this sentence can be seen in (\ref{ex:rel:tree_obj}).

\begin{exe}
  \ex\label{ex:rel:tree_obj} \Tree [.S 
    [.NP
      [.CP 
        [.S 
          [.NP N\\$\emptyset$ ]
          [.VP
            [.NP N\\$t$ ]
            V\\$t$
          ]
        ]
        C\\\orth{ye-serawt}
      ] 
      N\\\orth{buna}
    ]
    [.VP 
      [.AdvP Adv\\\orth{yet} ]
      V\\\orth{new}
    ]
  ] 
\end{exe}

\end{document}
